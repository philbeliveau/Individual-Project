% Options for packages loaded elsewhere
\PassOptionsToPackage{unicode}{hyperref}
\PassOptionsToPackage{hyphens}{url}
%
\documentclass[
]{article}
\usepackage{amsmath,amssymb}
\usepackage{lmodern}
\usepackage{iftex}
\ifPDFTeX
  \usepackage[T1]{fontenc}
  \usepackage[utf8]{inputenc}
  \usepackage{textcomp} % provide euro and other symbols
\else % if luatex or xetex
  \usepackage{unicode-math}
  \defaultfontfeatures{Scale=MatchLowercase}
  \defaultfontfeatures[\rmfamily]{Ligatures=TeX,Scale=1}
\fi
% Use upquote if available, for straight quotes in verbatim environments
\IfFileExists{upquote.sty}{\usepackage{upquote}}{}
\IfFileExists{microtype.sty}{% use microtype if available
  \usepackage[]{microtype}
  \UseMicrotypeSet[protrusion]{basicmath} % disable protrusion for tt fonts
}{}
\makeatletter
\@ifundefined{KOMAClassName}{% if non-KOMA class
  \IfFileExists{parskip.sty}{%
    \usepackage{parskip}
  }{% else
    \setlength{\parindent}{0pt}
    \setlength{\parskip}{6pt plus 2pt minus 1pt}}
}{% if KOMA class
  \KOMAoptions{parskip=half}}
\makeatother
\usepackage{xcolor}
\usepackage[margin=1in]{geometry}
\usepackage{color}
\usepackage{fancyvrb}
\newcommand{\VerbBar}{|}
\newcommand{\VERB}{\Verb[commandchars=\\\{\}]}
\DefineVerbatimEnvironment{Highlighting}{Verbatim}{commandchars=\\\{\}}
% Add ',fontsize=\small' for more characters per line
\usepackage{framed}
\definecolor{shadecolor}{RGB}{248,248,248}
\newenvironment{Shaded}{\begin{snugshade}}{\end{snugshade}}
\newcommand{\AlertTok}[1]{\textcolor[rgb]{0.94,0.16,0.16}{#1}}
\newcommand{\AnnotationTok}[1]{\textcolor[rgb]{0.56,0.35,0.01}{\textbf{\textit{#1}}}}
\newcommand{\AttributeTok}[1]{\textcolor[rgb]{0.77,0.63,0.00}{#1}}
\newcommand{\BaseNTok}[1]{\textcolor[rgb]{0.00,0.00,0.81}{#1}}
\newcommand{\BuiltInTok}[1]{#1}
\newcommand{\CharTok}[1]{\textcolor[rgb]{0.31,0.60,0.02}{#1}}
\newcommand{\CommentTok}[1]{\textcolor[rgb]{0.56,0.35,0.01}{\textit{#1}}}
\newcommand{\CommentVarTok}[1]{\textcolor[rgb]{0.56,0.35,0.01}{\textbf{\textit{#1}}}}
\newcommand{\ConstantTok}[1]{\textcolor[rgb]{0.00,0.00,0.00}{#1}}
\newcommand{\ControlFlowTok}[1]{\textcolor[rgb]{0.13,0.29,0.53}{\textbf{#1}}}
\newcommand{\DataTypeTok}[1]{\textcolor[rgb]{0.13,0.29,0.53}{#1}}
\newcommand{\DecValTok}[1]{\textcolor[rgb]{0.00,0.00,0.81}{#1}}
\newcommand{\DocumentationTok}[1]{\textcolor[rgb]{0.56,0.35,0.01}{\textbf{\textit{#1}}}}
\newcommand{\ErrorTok}[1]{\textcolor[rgb]{0.64,0.00,0.00}{\textbf{#1}}}
\newcommand{\ExtensionTok}[1]{#1}
\newcommand{\FloatTok}[1]{\textcolor[rgb]{0.00,0.00,0.81}{#1}}
\newcommand{\FunctionTok}[1]{\textcolor[rgb]{0.00,0.00,0.00}{#1}}
\newcommand{\ImportTok}[1]{#1}
\newcommand{\InformationTok}[1]{\textcolor[rgb]{0.56,0.35,0.01}{\textbf{\textit{#1}}}}
\newcommand{\KeywordTok}[1]{\textcolor[rgb]{0.13,0.29,0.53}{\textbf{#1}}}
\newcommand{\NormalTok}[1]{#1}
\newcommand{\OperatorTok}[1]{\textcolor[rgb]{0.81,0.36,0.00}{\textbf{#1}}}
\newcommand{\OtherTok}[1]{\textcolor[rgb]{0.56,0.35,0.01}{#1}}
\newcommand{\PreprocessorTok}[1]{\textcolor[rgb]{0.56,0.35,0.01}{\textit{#1}}}
\newcommand{\RegionMarkerTok}[1]{#1}
\newcommand{\SpecialCharTok}[1]{\textcolor[rgb]{0.00,0.00,0.00}{#1}}
\newcommand{\SpecialStringTok}[1]{\textcolor[rgb]{0.31,0.60,0.02}{#1}}
\newcommand{\StringTok}[1]{\textcolor[rgb]{0.31,0.60,0.02}{#1}}
\newcommand{\VariableTok}[1]{\textcolor[rgb]{0.00,0.00,0.00}{#1}}
\newcommand{\VerbatimStringTok}[1]{\textcolor[rgb]{0.31,0.60,0.02}{#1}}
\newcommand{\WarningTok}[1]{\textcolor[rgb]{0.56,0.35,0.01}{\textbf{\textit{#1}}}}
\usepackage{graphicx}
\makeatletter
\def\maxwidth{\ifdim\Gin@nat@width>\linewidth\linewidth\else\Gin@nat@width\fi}
\def\maxheight{\ifdim\Gin@nat@height>\textheight\textheight\else\Gin@nat@height\fi}
\makeatother
% Scale images if necessary, so that they will not overflow the page
% margins by default, and it is still possible to overwrite the defaults
% using explicit options in \includegraphics[width, height, ...]{}
\setkeys{Gin}{width=\maxwidth,height=\maxheight,keepaspectratio}
% Set default figure placement to htbp
\makeatletter
\def\fps@figure{htbp}
\makeatother
\setlength{\emergencystretch}{3em} % prevent overfull lines
\providecommand{\tightlist}{%
  \setlength{\itemsep}{0pt}\setlength{\parskip}{0pt}}
\setcounter{secnumdepth}{-\maxdimen} % remove section numbering
\usepackage{booktabs}
\usepackage{longtable}
\usepackage{array}
\usepackage{multirow}
\usepackage{wrapfig}
\usepackage{float}
\usepackage{colortbl}
\usepackage{pdflscape}
\usepackage{tabu}
\usepackage{threeparttable}
\usepackage{threeparttablex}
\usepackage[normalem]{ulem}
\usepackage{makecell}
\usepackage{xcolor}
\ifLuaTeX
  \usepackage{selnolig}  % disable illegal ligatures
\fi
\IfFileExists{bookmark.sty}{\usepackage{bookmark}}{\usepackage{hyperref}}
\IfFileExists{xurl.sty}{\usepackage{xurl}}{} % add URL line breaks if available
\urlstyle{same} % disable monospaced font for URLs
\hypersetup{
  pdftitle={Question 1 \& Question 2},
  hidelinks,
  pdfcreator={LaTeX via pandoc}}

\title{Question 1 \& Question 2}
\author{}
\date{\vspace{-2.5em}}

\begin{document}
\maketitle

\hypertarget{question-1}{%
\subsection{Question 1}\label{question-1}}

\hypertarget{a.-perform-an-exploratory-data-analysis-eda-to-extract-different-data-characteristics.-perform-necessary-manipulations-e.g.-treatment-of-outliers.-use-95th-percentile-for-outlier-capping-if-needed.-comment.}{%
\subparagraph{1.a. Perform an exploratory data analysis (EDA) to extract
different data characteristics. Perform necessary manipulations (e.g.,
treatment of outliers). Use 95th percentile for outlier capping, if
needed.
Comment.}\label{a.-perform-an-exploratory-data-analysis-eda-to-extract-different-data-characteristics.-perform-necessary-manipulations-e.g.-treatment-of-outliers.-use-95th-percentile-for-outlier-capping-if-needed.-comment.}}

\begin{Shaded}
\begin{Highlighting}[]
\FunctionTok{library}\NormalTok{(ggplot2)}
\FunctionTok{library}\NormalTok{(dplyr)}
\end{Highlighting}
\end{Shaded}

\begin{verbatim}
## 
## Attaching package: 'dplyr'
\end{verbatim}

\begin{verbatim}
## The following objects are masked from 'package:stats':
## 
##     filter, lag
\end{verbatim}

\begin{verbatim}
## The following objects are masked from 'package:base':
## 
##     intersect, setdiff, setequal, union
\end{verbatim}

\begin{Shaded}
\begin{Highlighting}[]
\FunctionTok{library}\NormalTok{(magrittr)}
\end{Highlighting}
\end{Shaded}

First, in order to make the problem/dataset more interesting, we
categorize the alcohol variable (which is initially numerical) into
three levels: low, medium, and high. The separation rule is as follows:

\begin{itemize}
\item
  Low (\texttt{levels.alcohol} == 1) when alcohol is below \(10.5 \%\).
\item
  Medium (\texttt{levels.alcohol} == 2) when alcohol is below \(12 \%\).
\item
  High (\texttt{levels.alcohol} == 3) when alcohol is above \(12 \%\).
\end{itemize}

\begin{Shaded}
\begin{Highlighting}[]
\NormalTok{wineDataset }\OtherTok{\textless{}{-}} \FunctionTok{read.csv}\NormalTok{(}\StringTok{"/Users/philippebeliveau/Downloads/winequality{-}red.csv"}\NormalTok{)}

\NormalTok{alcohol\_levels }\OtherTok{\textless{}{-}}\NormalTok{ wineDataset }\SpecialCharTok{\%\textgreater{}\%} \FunctionTok{select}\NormalTok{(alcohol) }\SpecialCharTok{\%\textgreater{}\%} \FunctionTok{mutate}\NormalTok{(}\StringTok{"Low\_alcohol"} \OtherTok{=}\NormalTok{ alcohol}\SpecialCharTok{\textless{}=}\FloatTok{10.5}\NormalTok{, }\StringTok{"normal\_alcohol"} \OtherTok{=}\NormalTok{ alcohol}\SpecialCharTok{\textgreater{}}\FloatTok{10.5} \SpecialCharTok{\&}\NormalTok{ alcohol}\SpecialCharTok{\textless{}=}\DecValTok{12}\NormalTok{, }
                                                      \StringTok{"high\_alcohol"}\OtherTok{=}\NormalTok{ alcohol}\SpecialCharTok{\textgreater{}}\DecValTok{12}\NormalTok{)}

\NormalTok{alcohol\_levels}\SpecialCharTok{$}\NormalTok{Low\_alcohol }\OtherTok{\textless{}{-}} \FunctionTok{as.integer}\NormalTok{(}\FunctionTok{as.logical}\NormalTok{(alcohol\_levels}\SpecialCharTok{$}\NormalTok{Low\_alcohol))}
\NormalTok{alcohol\_levels}\SpecialCharTok{$}\NormalTok{normal\_alcohol }\OtherTok{\textless{}{-}} \FunctionTok{as.integer}\NormalTok{(}\FunctionTok{as.logical}\NormalTok{(alcohol\_levels}\SpecialCharTok{$}\NormalTok{normal\_alcohol))}
\NormalTok{alcohol\_levels}\SpecialCharTok{$}\NormalTok{high\_alcohol }\OtherTok{\textless{}{-}} \FunctionTok{as.integer}\NormalTok{(}\FunctionTok{as.logical}\NormalTok{(alcohol\_levels}\SpecialCharTok{$}\NormalTok{high\_alcohol))}
\NormalTok{alcohol\_levels }\OtherTok{\textless{}{-}}\NormalTok{ alcohol\_levels[, }\SpecialCharTok{{-}}\DecValTok{1}\NormalTok{]}

\NormalTok{wineDataset }\OtherTok{\textless{}{-}} \FunctionTok{cbind}\NormalTok{(wineDataset, alcohol\_levels)}

\NormalTok{wine\_t }\OtherTok{\textless{}{-}}\NormalTok{ wineDataset }\SpecialCharTok{\%\textgreater{}\%} \FunctionTok{mutate}\NormalTok{(}\AttributeTok{levels.alcohol=}
                            \FunctionTok{case\_when}\NormalTok{(Low\_alcohol}\SpecialCharTok{==}\DecValTok{1}\SpecialCharTok{\textasciitilde{}}\DecValTok{1}\NormalTok{, normal\_alcohol}\SpecialCharTok{==}\DecValTok{1}\SpecialCharTok{\textasciitilde{}}\DecValTok{2}\NormalTok{, high\_alcohol}\SpecialCharTok{==}\DecValTok{1}\SpecialCharTok{\textasciitilde{}}\DecValTok{3}\NormalTok{))}

\NormalTok{wineDataset }\OtherTok{\textless{}{-}}\NormalTok{ wine\_t[, }\FunctionTok{c}\NormalTok{(}\DecValTok{1}\SpecialCharTok{:}\DecValTok{10}\NormalTok{, }\DecValTok{12}\NormalTok{, }\DecValTok{16}\NormalTok{)]}
\end{Highlighting}
\end{Shaded}

We'll start off by exploring the data\ldots{}

\begin{Shaded}
\begin{Highlighting}[]
\NormalTok{wineDataset }\OtherTok{\textless{}{-}}\NormalTok{ wineDataset[}\FunctionTok{complete.cases}\NormalTok{(wineDataset), ]}

\FunctionTok{str}\NormalTok{(wineDataset)}
\end{Highlighting}
\end{Shaded}

\begin{verbatim}
## 'data.frame':    1599 obs. of  12 variables:
##  $ fixed.acidity       : num  7.4 7.8 7.8 11.2 7.4 7.4 7.9 7.3 7.8 7.5 ...
##  $ volatile.acidity    : num  0.7 0.88 0.76 0.28 0.7 0.66 0.6 0.65 0.58 0.5 ...
##  $ citric.acid         : num  0 0 0.04 0.56 0 0 0.06 0 0.02 0.36 ...
##  $ residual.sugar      : num  1.9 2.6 2.3 1.9 1.9 1.8 1.6 1.2 2 6.1 ...
##  $ chlorides           : num  0.076 0.098 0.092 0.075 0.076 0.075 0.069 0.065 0.073 0.071 ...
##  $ free.sulfur.dioxide : num  11 25 15 17 11 13 15 15 9 17 ...
##  $ total.sulfur.dioxide: num  34 67 54 60 34 40 59 21 18 102 ...
##  $ density             : num  0.998 0.997 0.997 0.998 0.998 ...
##  $ pH                  : num  3.51 3.2 3.26 3.16 3.51 3.51 3.3 3.39 3.36 3.35 ...
##  $ sulphates           : num  0.56 0.68 0.65 0.58 0.56 0.56 0.46 0.47 0.57 0.8 ...
##  $ quality             : int  5 5 5 6 5 5 5 7 7 5 ...
##  $ levels.alcohol      : num  1 1 1 1 1 1 1 1 1 1 ...
\end{verbatim}

\begin{Shaded}
\begin{Highlighting}[]
\FunctionTok{head}\NormalTok{(wineDataset)}
\end{Highlighting}
\end{Shaded}

\begin{verbatim}
##   fixed.acidity volatile.acidity citric.acid residual.sugar chlorides
## 1           7.4             0.70        0.00            1.9     0.076
## 2           7.8             0.88        0.00            2.6     0.098
## 3           7.8             0.76        0.04            2.3     0.092
## 4          11.2             0.28        0.56            1.9     0.075
## 5           7.4             0.70        0.00            1.9     0.076
## 6           7.4             0.66        0.00            1.8     0.075
##   free.sulfur.dioxide total.sulfur.dioxide density   pH sulphates quality
## 1                  11                   34  0.9978 3.51      0.56       5
## 2                  25                   67  0.9968 3.20      0.68       5
## 3                  15                   54  0.9970 3.26      0.65       5
## 4                  17                   60  0.9980 3.16      0.58       6
## 5                  11                   34  0.9978 3.51      0.56       5
## 6                  13                   40  0.9978 3.51      0.56       5
##   levels.alcohol
## 1              1
## 2              1
## 3              1
## 4              1
## 5              1
## 6              1
\end{verbatim}

\begin{Shaded}
\begin{Highlighting}[]
\FunctionTok{colSums}\NormalTok{(}\FunctionTok{is.na}\NormalTok{(wineDataset))}
\end{Highlighting}
\end{Shaded}

\begin{verbatim}
##        fixed.acidity     volatile.acidity          citric.acid 
##                    0                    0                    0 
##       residual.sugar            chlorides  free.sulfur.dioxide 
##                    0                    0                    0 
## total.sulfur.dioxide              density                   pH 
##                    0                    0                    0 
##            sulphates              quality       levels.alcohol 
##                    0                    0                    0
\end{verbatim}

\textbf{Comment:} There are several observations:

\begin{itemize}
\item
  The dataset contains 1599 observations.
\item
  The dataset has 11 explanatory variables and doesn't have any NA
  values.
\item
  All variables are continuous variables, except the level of alcohol,
  which we will treat as a categorical variable with three distinct
  categories.
\end{itemize}

\hypertarget{a.i-descriptive-statistics}{%
\subparagraph{1.a.i Descriptive
Statistics}\label{a.i-descriptive-statistics}}

Some descriptive statistics:

\begin{Shaded}
\begin{Highlighting}[]
\NormalTok{summary}\OtherTok{\textless{}{-}}\FunctionTok{sapply}\NormalTok{(wineDataset,}\ControlFlowTok{function}\NormalTok{(x) }\FunctionTok{c}\NormalTok{(}\FunctionTok{mean}\NormalTok{(x),}\FunctionTok{sd}\NormalTok{(x),}\FunctionTok{min}\NormalTok{(x),}\FunctionTok{max}\NormalTok{(x),}\FunctionTok{length}\NormalTok{(x)))}
\FunctionTok{row.names}\NormalTok{(summary)}\OtherTok{\textless{}{-}}\FunctionTok{c}\NormalTok{(}\StringTok{"mean"}\NormalTok{,}\StringTok{"sd"}\NormalTok{,}\StringTok{"min"}\NormalTok{,}\StringTok{"max"}\NormalTok{,}\StringTok{"n"}\NormalTok{)}
\NormalTok{summary}
\end{Highlighting}
\end{Shaded}

\begin{verbatim}
##      fixed.acidity volatile.acidity  citric.acid residual.sugar    chlorides
## mean      8.319637        0.5278205    0.2709756       2.538806 8.746654e-02
## sd        1.741096        0.1790597    0.1948011       1.409928 4.706530e-02
## min       4.600000        0.1200000    0.0000000       0.900000 1.200000e-02
## max      15.900000        1.5800000    1.0000000      15.500000 6.110000e-01
## n      1599.000000     1599.0000000 1599.0000000    1599.000000 1.599000e+03
##      free.sulfur.dioxide total.sulfur.dioxide      density           pH
## mean            15.87492             46.46779 9.967467e-01    3.3111132
## sd              10.46016             32.89532 1.887334e-03    0.1543865
## min              1.00000              6.00000 9.900700e-01    2.7400000
## max             72.00000            289.00000 1.003690e+00    4.0100000
## n             1599.00000           1599.00000 1.599000e+03 1599.0000000
##         sulphates      quality levels.alcohol
## mean    0.6581488    5.6360225      1.4734209
## sd      0.1695070    0.8075694      0.6526256
## min     0.3300000    3.0000000      1.0000000
## max     2.0000000    8.0000000      3.0000000
## n    1599.0000000 1599.0000000   1599.0000000
\end{verbatim}

\textbf{Comment:} We notice that variables have different scales.
Certain variables such as \texttt{free.sulfur.dioxide} and
\texttt{total.sulfur.dioxide} have a large standard deviation, whereas
others like \texttt{density} have very small standard deviation.

We are looking for features that hold significant amount of information
about our response variable. Thus, we are especially looking for
features with high standard deviation. free.sulfur.dioxide and
total.sulfur.dioxide seems to be good candidates for helping us explain
the quality of the wine.

\hypertarget{a.ii-distribution-of-the-variables}{%
\subparagraph{1.a.ii Distribution of the
variables:}\label{a.ii-distribution-of-the-variables}}

\hypertarget{histogram-of-the-explanatory-variables}{%
\subparagraph{Histogram of the explanatory
variables:}\label{histogram-of-the-explanatory-variables}}

\begin{Shaded}
\begin{Highlighting}[]
\CommentTok{\# Histograms}
\FunctionTok{par}\NormalTok{(}\AttributeTok{mfrow=}\FunctionTok{c}\NormalTok{(}\DecValTok{2}\NormalTok{,}\DecValTok{2}\NormalTok{))}
\FunctionTok{hist}\NormalTok{(wineDataset}\SpecialCharTok{$}\NormalTok{fixed.acidity,}\AttributeTok{xlab=}\StringTok{"fixed acidity"}\NormalTok{,}\AttributeTok{main=}\StringTok{"Histogram"}\NormalTok{)}
\FunctionTok{hist}\NormalTok{(wineDataset}\SpecialCharTok{$}\NormalTok{volatile.acidity,}\AttributeTok{xlab=}\StringTok{"volatile acidity"}\NormalTok{,}\AttributeTok{main=}\StringTok{"Histogram"}\NormalTok{)}
\FunctionTok{hist}\NormalTok{(wineDataset}\SpecialCharTok{$}\NormalTok{citric.acid,}\AttributeTok{xlab=}\StringTok{"citric acid"}\NormalTok{,}\AttributeTok{main=}\StringTok{"Histogram"}\NormalTok{)}
\FunctionTok{hist}\NormalTok{(wineDataset}\SpecialCharTok{$}\NormalTok{residual.sugar,}\AttributeTok{xlab=}\StringTok{"residual sugar"}\NormalTok{,}\AttributeTok{main=}\StringTok{"Histogram"}\NormalTok{)}
\end{Highlighting}
\end{Shaded}

\includegraphics{Yenal_Arora_MATH-60604A_Final_Project_Solutions_files/figure-latex/unnamed-chunk-5-1.pdf}

\begin{Shaded}
\begin{Highlighting}[]
\FunctionTok{par}\NormalTok{(}\AttributeTok{mfrow=}\FunctionTok{c}\NormalTok{(}\DecValTok{2}\NormalTok{,}\DecValTok{2}\NormalTok{))}
\FunctionTok{hist}\NormalTok{(wineDataset}\SpecialCharTok{$}\NormalTok{chlorides,}\AttributeTok{xlab=}\StringTok{"chlorides"}\NormalTok{,}\AttributeTok{main=}\StringTok{"Histogram"}\NormalTok{)}
\FunctionTok{hist}\NormalTok{(wineDataset}\SpecialCharTok{$}\NormalTok{free.sulfur.dioxide,}\AttributeTok{xlab=}\StringTok{"free.sulfur.dioxide"}\NormalTok{,}\AttributeTok{main=}\StringTok{"Histogram"}\NormalTok{)}
\FunctionTok{hist}\NormalTok{(wineDataset}\SpecialCharTok{$}\NormalTok{total.sulfur.dioxide,}\AttributeTok{xlab=}\StringTok{"total.sulfur.dioxide"}\NormalTok{,}\AttributeTok{main=}\StringTok{"Histogram"}\NormalTok{)}
\FunctionTok{hist}\NormalTok{(wineDataset}\SpecialCharTok{$}\NormalTok{density,}\AttributeTok{xlab=}\StringTok{"density"}\NormalTok{,}\AttributeTok{main=}\StringTok{"Histogram"}\NormalTok{)}
\end{Highlighting}
\end{Shaded}

\includegraphics{Yenal_Arora_MATH-60604A_Final_Project_Solutions_files/figure-latex/unnamed-chunk-6-1.pdf}

\begin{Shaded}
\begin{Highlighting}[]
\FunctionTok{par}\NormalTok{(}\AttributeTok{mfrow=}\FunctionTok{c}\NormalTok{(}\DecValTok{2}\NormalTok{,}\DecValTok{2}\NormalTok{))}
\FunctionTok{hist}\NormalTok{(wineDataset}\SpecialCharTok{$}\NormalTok{pH,}\AttributeTok{xlab=}\StringTok{"pH"}\NormalTok{,}\AttributeTok{main=}\StringTok{"Histogram"}\NormalTok{)}
\FunctionTok{hist}\NormalTok{(wineDataset}\SpecialCharTok{$}\NormalTok{sulphates,}\AttributeTok{xlab=}\StringTok{"sulphates"}\NormalTok{,}\AttributeTok{main=}\StringTok{"Histogram"}\NormalTok{)}
\FunctionTok{hist}\NormalTok{(wineDataset}\SpecialCharTok{$}\NormalTok{levels.alcohol,}\AttributeTok{xlab=}\StringTok{"alcohol"}\NormalTok{,}\AttributeTok{main=}\StringTok{"Histogram"}\NormalTok{)}
\end{Highlighting}
\end{Shaded}

\includegraphics{Yenal_Arora_MATH-60604A_Final_Project_Solutions_files/figure-latex/unnamed-chunk-7-1.pdf}

\hypertarget{histogram-of-the-response-variable-quality}{%
\subparagraph{\texorpdfstring{Histogram of the response variable
(\emph{quality}):}{Histogram of the response variable (quality):}}\label{histogram-of-the-response-variable-quality}}

\begin{Shaded}
\begin{Highlighting}[]
\FunctionTok{hist}\NormalTok{(wineDataset}\SpecialCharTok{$}\NormalTok{quality,}\AttributeTok{xlab=}\StringTok{"quality"}\NormalTok{,}\AttributeTok{main=}\StringTok{"Histogram"}\NormalTok{)}
\end{Highlighting}
\end{Shaded}

\includegraphics{Yenal_Arora_MATH-60604A_Final_Project_Solutions_files/figure-latex/unnamed-chunk-8-1.pdf}

\textbf{Comment:} Some observations include:

\begin{itemize}
\item
  We see that the majority of our explanatory variables are slightly
  skewed to the right. We could then consider using a different
  correlation coefficient that doesn't require normality between the
  explanatory variable to assess the correlation between them.
\item
  We also see that our response variable is relatively normally
  distributed. It is expected; low and high quality wines are less
  observed, while average quality wines are the most frequent. There are
  more than 68.9\% of the observations that are within one standard
  deviation, thus the distribution is not perfectly normally
  distributed.
\item
  Another interesting observation includes the imbalance of the levels
  of alcohol. We see that most wines have a low level of alcohol. It
  could be interesting to look into more details at the characteristic
  that compose higher level of alcohol.
\end{itemize}

\hypertarget{a.iii-scatterplots}{%
\subparagraph{1.a.iii Scatterplots}\label{a.iii-scatterplots}}

We can also examine the relationship between the response variable and
the explanatory variables using scatterplots.

\hypertarget{fixed-acidity-and-volatile-acidity-vs.-quality-scatter-plots}{%
\subparagraph{Fixed acidity and volatile acidity vs.~quality scatter
plots}\label{fixed-acidity-and-volatile-acidity-vs.-quality-scatter-plots}}

\begin{Shaded}
\begin{Highlighting}[]
\FunctionTok{par}\NormalTok{(}\AttributeTok{mfrow=}\FunctionTok{c}\NormalTok{(}\DecValTok{1}\NormalTok{,}\DecValTok{2}\NormalTok{))}
\CommentTok{\# quality, fixed.acidity}
\FunctionTok{plot}\NormalTok{(quality}\SpecialCharTok{\textasciitilde{}}\NormalTok{fixed.acidity,}\AttributeTok{xlab=}\StringTok{"fixed acidity"}\NormalTok{,}\AttributeTok{ylab=}\StringTok{"quality"}\NormalTok{,}\AttributeTok{data=}\NormalTok{wineDataset,}\AttributeTok{lwd=}\FloatTok{1.5}\NormalTok{)}
\FunctionTok{abline}\NormalTok{(}\FunctionTok{lm}\NormalTok{(quality}\SpecialCharTok{\textasciitilde{}}\NormalTok{fixed.acidity,}\AttributeTok{data=}\NormalTok{wineDataset),}\AttributeTok{lwd=}\FloatTok{1.5}\NormalTok{)}

\CommentTok{\# quality, volatile.acidity}
\FunctionTok{plot}\NormalTok{(quality}\SpecialCharTok{\textasciitilde{}}\NormalTok{volatile.acidity,}\AttributeTok{xlab=}\StringTok{"volatile acidity"}\NormalTok{,}\AttributeTok{ylab=}\StringTok{"quality"}\NormalTok{,}\AttributeTok{data=}\NormalTok{wineDataset,}\AttributeTok{lwd=}\FloatTok{1.5}\NormalTok{)}
\FunctionTok{abline}\NormalTok{(}\FunctionTok{lm}\NormalTok{(quality}\SpecialCharTok{\textasciitilde{}}\NormalTok{volatile.acidity,}\AttributeTok{data=}\NormalTok{wineDataset),}\AttributeTok{lwd=}\FloatTok{1.5}\NormalTok{)}
\end{Highlighting}
\end{Shaded}

\includegraphics{Yenal_Arora_MATH-60604A_Final_Project_Solutions_files/figure-latex/unnamed-chunk-9-1.pdf}

\hypertarget{citric-acid-and-ph-vs.-quality-scatter-plots}{%
\subparagraph{citric acid and pH vs.~quality scatter
plots}\label{citric-acid-and-ph-vs.-quality-scatter-plots}}

\begin{Shaded}
\begin{Highlighting}[]
\FunctionTok{par}\NormalTok{(}\AttributeTok{mfrow=}\FunctionTok{c}\NormalTok{(}\DecValTok{1}\NormalTok{,}\DecValTok{2}\NormalTok{))}
\CommentTok{\# quality, citric.acid}
\FunctionTok{plot}\NormalTok{(quality}\SpecialCharTok{\textasciitilde{}}\NormalTok{citric.acid,}\AttributeTok{xlab=}\StringTok{"citric acid"}\NormalTok{,}\AttributeTok{ylab=}\StringTok{"quality"}\NormalTok{,}\AttributeTok{data=}\NormalTok{wineDataset,}\AttributeTok{lwd=}\FloatTok{1.5}\NormalTok{)}
\FunctionTok{abline}\NormalTok{(}\FunctionTok{lm}\NormalTok{(quality}\SpecialCharTok{\textasciitilde{}}\NormalTok{citric.acid,}\AttributeTok{data=}\NormalTok{wineDataset),}\AttributeTok{lwd=}\FloatTok{1.5}\NormalTok{)}

\CommentTok{\# quality, pH}
\FunctionTok{plot}\NormalTok{(quality}\SpecialCharTok{\textasciitilde{}}\NormalTok{pH,}\AttributeTok{xlab=}\StringTok{"pH"}\NormalTok{,}\AttributeTok{ylab=}\StringTok{"quality"}\NormalTok{,}\AttributeTok{data=}\NormalTok{wineDataset,}\AttributeTok{lwd=}\FloatTok{1.5}\NormalTok{)}
\FunctionTok{abline}\NormalTok{(}\FunctionTok{lm}\NormalTok{(quality}\SpecialCharTok{\textasciitilde{}}\NormalTok{pH,}\AttributeTok{data=}\NormalTok{wineDataset),}\AttributeTok{lwd=}\FloatTok{1.5}\NormalTok{)}
\end{Highlighting}
\end{Shaded}

\includegraphics{Yenal_Arora_MATH-60604A_Final_Project_Solutions_files/figure-latex/unnamed-chunk-10-1.pdf}

\hypertarget{residual-sugar-and-chlorides-vs.-quality-scatter-plots}{%
\subparagraph{residual sugar and chlorides vs.~quality scatter
plots}\label{residual-sugar-and-chlorides-vs.-quality-scatter-plots}}

\begin{Shaded}
\begin{Highlighting}[]
\FunctionTok{par}\NormalTok{(}\AttributeTok{mfrow=}\FunctionTok{c}\NormalTok{(}\DecValTok{1}\NormalTok{,}\DecValTok{2}\NormalTok{))}
\CommentTok{\# quality, residual.sugar}
\FunctionTok{plot}\NormalTok{(quality}\SpecialCharTok{\textasciitilde{}}\NormalTok{residual.sugar,}\AttributeTok{xlab=}\StringTok{"residual sugar"}\NormalTok{,}\AttributeTok{ylab=}\StringTok{"quality"}\NormalTok{,}\AttributeTok{data=}\NormalTok{wineDataset,}\AttributeTok{lwd=}\FloatTok{1.5}\NormalTok{)}
\FunctionTok{abline}\NormalTok{(}\FunctionTok{lm}\NormalTok{(quality}\SpecialCharTok{\textasciitilde{}}\NormalTok{residual.sugar,}\AttributeTok{data=}\NormalTok{wineDataset),}\AttributeTok{lwd=}\FloatTok{1.5}\NormalTok{)}

\CommentTok{\# quality, chlorides}
\FunctionTok{plot}\NormalTok{(quality}\SpecialCharTok{\textasciitilde{}}\NormalTok{chlorides,}\AttributeTok{xlab=}\StringTok{"chlorides"}\NormalTok{,}\AttributeTok{ylab=}\StringTok{"quality"}\NormalTok{,}\AttributeTok{data=}\NormalTok{wineDataset,}\AttributeTok{lwd=}\FloatTok{1.5}\NormalTok{)}
\FunctionTok{abline}\NormalTok{(}\FunctionTok{lm}\NormalTok{(quality}\SpecialCharTok{\textasciitilde{}}\NormalTok{chlorides,}\AttributeTok{data=}\NormalTok{wineDataset),}\AttributeTok{lwd=}\FloatTok{1.5}\NormalTok{)}
\end{Highlighting}
\end{Shaded}

\includegraphics{Yenal_Arora_MATH-60604A_Final_Project_Solutions_files/figure-latex/unnamed-chunk-11-1.pdf}

\hypertarget{free.sulfur.dioxide-and-total.sulfur.dioxide-vs.-quality-scatter-plots}{%
\subparagraph{free.sulfur.dioxide and total.sulfur.dioxide vs.~quality
scatter
plots}\label{free.sulfur.dioxide-and-total.sulfur.dioxide-vs.-quality-scatter-plots}}

\begin{Shaded}
\begin{Highlighting}[]
\FunctionTok{par}\NormalTok{(}\AttributeTok{mfrow=}\FunctionTok{c}\NormalTok{(}\DecValTok{1}\NormalTok{,}\DecValTok{2}\NormalTok{))}
\CommentTok{\# quality, free.sulfur.dioxide}
\FunctionTok{plot}\NormalTok{(quality}\SpecialCharTok{\textasciitilde{}}\NormalTok{free.sulfur.dioxide,}\AttributeTok{xlab=}\StringTok{"free sulfur dioxide"}\NormalTok{,}\AttributeTok{ylab=}\StringTok{"quality"}\NormalTok{,}\AttributeTok{data=}\NormalTok{wineDataset,}\AttributeTok{lwd=}\FloatTok{1.5}\NormalTok{)}
\FunctionTok{abline}\NormalTok{(}\FunctionTok{lm}\NormalTok{(quality}\SpecialCharTok{\textasciitilde{}}\NormalTok{free.sulfur.dioxide,}\AttributeTok{data=}\NormalTok{wineDataset),}\AttributeTok{lwd=}\FloatTok{1.5}\NormalTok{)}

\CommentTok{\# quality, total.sulfur.dioxide}
\FunctionTok{plot}\NormalTok{(quality}\SpecialCharTok{\textasciitilde{}}\NormalTok{total.sulfur.dioxide,}\AttributeTok{xlab=}\StringTok{"total sulfur dioxide"}\NormalTok{,}\AttributeTok{ylab=}\StringTok{"quality"}\NormalTok{,}\AttributeTok{data=}\NormalTok{wineDataset,}\AttributeTok{lwd=}\FloatTok{1.5}\NormalTok{)}
\FunctionTok{abline}\NormalTok{(}\FunctionTok{lm}\NormalTok{(quality}\SpecialCharTok{\textasciitilde{}}\NormalTok{total.sulfur.dioxide,}\AttributeTok{data=}\NormalTok{wineDataset),}\AttributeTok{lwd=}\FloatTok{1.5}\NormalTok{)}
\end{Highlighting}
\end{Shaded}

\includegraphics{Yenal_Arora_MATH-60604A_Final_Project_Solutions_files/figure-latex/unnamed-chunk-12-1.pdf}

\hypertarget{density-and-alcohol-vs.-quality-scatter-plots}{%
\subparagraph{density and alcohol vs.~quality scatter
plots}\label{density-and-alcohol-vs.-quality-scatter-plots}}

\begin{Shaded}
\begin{Highlighting}[]
\FunctionTok{par}\NormalTok{(}\AttributeTok{mfrow=}\FunctionTok{c}\NormalTok{(}\DecValTok{1}\NormalTok{,}\DecValTok{2}\NormalTok{))}
\CommentTok{\# quality, density}
\FunctionTok{plot}\NormalTok{(quality}\SpecialCharTok{\textasciitilde{}}\NormalTok{density,}\AttributeTok{xlab=}\StringTok{"density"}\NormalTok{,}\AttributeTok{ylab=}\StringTok{"quality"}\NormalTok{,}\AttributeTok{data=}\NormalTok{wineDataset,}\AttributeTok{lwd=}\FloatTok{1.5}\NormalTok{)}
\FunctionTok{abline}\NormalTok{(}\FunctionTok{lm}\NormalTok{(quality}\SpecialCharTok{\textasciitilde{}}\NormalTok{density,}\AttributeTok{data=}\NormalTok{wineDataset),}\AttributeTok{lwd=}\FloatTok{1.5}\NormalTok{)}
\end{Highlighting}
\end{Shaded}

\includegraphics{Yenal_Arora_MATH-60604A_Final_Project_Solutions_files/figure-latex/unnamed-chunk-13-1.pdf}

\hypertarget{sulphates-vs.-quality-scatter-plot}{%
\subparagraph{sulphates vs.~quality scatter
plot}\label{sulphates-vs.-quality-scatter-plot}}

\begin{Shaded}
\begin{Highlighting}[]
\FunctionTok{par}\NormalTok{(}\AttributeTok{mfrow=}\FunctionTok{c}\NormalTok{(}\DecValTok{1}\NormalTok{,}\DecValTok{2}\NormalTok{))}
\CommentTok{\# quality, sulphates}
\FunctionTok{plot}\NormalTok{(quality}\SpecialCharTok{\textasciitilde{}}\NormalTok{sulphates,}\AttributeTok{xlab=}\StringTok{"sulphates"}\NormalTok{,}\AttributeTok{ylab=}\StringTok{"quality"}\NormalTok{,}\AttributeTok{data=}\NormalTok{wineDataset,}\AttributeTok{lwd=}\FloatTok{1.5}\NormalTok{)}
\FunctionTok{abline}\NormalTok{(}\FunctionTok{lm}\NormalTok{(quality}\SpecialCharTok{\textasciitilde{}}\NormalTok{sulphates,}\AttributeTok{data=}\NormalTok{wineDataset),}\AttributeTok{lwd=}\FloatTok{1.5}\NormalTok{)}
\end{Highlighting}
\end{Shaded}

\includegraphics{Yenal_Arora_MATH-60604A_Final_Project_Solutions_files/figure-latex/unnamed-chunk-14-1.pdf}

\textbf{Comment:} Highlights from this inspection include:

\begin{itemize}
\item
  There seems to have a strong negative correlation between the volatile
  acidity and the quality of the wine. Meaning that low quality wine
  have significantly higher volatile acidity level than good quality
  wine.
\item
  We also see that having higher levels of chloride seems to be
  associated with lower quality of wine.
\item
  The graph of quality and density seems interesting, as the value of
  density seems to be able to distinguish between low quality wine (3,5)
  to decent and very good wine (6,8). This make us believe that the
  level of density might be a good candidate to distinguish the wine
  quality.
\end{itemize}

\textbf{Conclusion:} These graphs only tell a limited story, as they
examine the relationship between quality and one explanatory variable at
a time.By looking at these graphs, it is not possible to deduct how
these variables effect quality of wine together.

\hypertarget{a.iv.-treatment-of-outliers}{%
\subparagraph{1.a.iv. Treatment of
outliers}\label{a.iv.-treatment-of-outliers}}

From the EDA, we observe that some variables might have outliers (e.g.,
total sulfur dioxide). It is therefore critical to treat them before
modeling. We will use the common method of capping the lowest and the
highest \(5\%\) of the observations for all the numerical variables.

\begin{Shaded}
\begin{Highlighting}[]
\FunctionTok{library}\NormalTok{(dlookr)}
\end{Highlighting}
\end{Shaded}

\begin{verbatim}
## Warning in !is.null(rmarkdown::metadata$output) && rmarkdown::metadata$output
## %in% : 'length(x) = 2 > 1' in coercion to 'logical(1)'
\end{verbatim}

\begin{verbatim}
## 
## Attaching package: 'dlookr'
\end{verbatim}

\begin{verbatim}
## The following object is masked from 'package:magrittr':
## 
##     extract
\end{verbatim}

\begin{verbatim}
## The following object is masked from 'package:base':
## 
##     transform
\end{verbatim}

\begin{Shaded}
\begin{Highlighting}[]
\NormalTok{wineDataset}\SpecialCharTok{$}\NormalTok{fixed.acidity }\OtherTok{\textless{}{-}} \FunctionTok{imputate\_outlier}\NormalTok{(wineDataset, fixed.acidity, }\AttributeTok{method =} \StringTok{"capping"}\NormalTok{)}
\NormalTok{wineDataset}\SpecialCharTok{$}\NormalTok{volatile.acidity }\OtherTok{\textless{}{-}} \FunctionTok{imputate\_outlier}\NormalTok{(wineDataset, volatile.acidity, }\AttributeTok{method =} \StringTok{"capping"}\NormalTok{)}
\NormalTok{wineDataset}\SpecialCharTok{$}\NormalTok{citric.acid }\OtherTok{\textless{}{-}} \FunctionTok{imputate\_outlier}\NormalTok{(wineDataset, citric.acid, }\AttributeTok{method =} \StringTok{"capping"}\NormalTok{)}
\NormalTok{wineDataset}\SpecialCharTok{$}\NormalTok{pH }\OtherTok{\textless{}{-}} \FunctionTok{imputate\_outlier}\NormalTok{(wineDataset, pH, }\AttributeTok{method =} \StringTok{"capping"}\NormalTok{)}

\NormalTok{wineDataset}\SpecialCharTok{$}\NormalTok{residual.sugar }\OtherTok{\textless{}{-}} \FunctionTok{imputate\_outlier}\NormalTok{(wineDataset, residual.sugar, }\AttributeTok{method =} \StringTok{"capping"}\NormalTok{)}
\NormalTok{wineDataset}\SpecialCharTok{$}\NormalTok{chlorides }\OtherTok{\textless{}{-}} \FunctionTok{imputate\_outlier}\NormalTok{(wineDataset, chlorides, }\AttributeTok{method =} \StringTok{"capping"}\NormalTok{)}
\NormalTok{wineDataset}\SpecialCharTok{$}\NormalTok{free.sulfur.dioxide }\OtherTok{\textless{}{-}} \FunctionTok{imputate\_outlier}\NormalTok{(wineDataset, free.sulfur.dioxide, }\AttributeTok{method =} \StringTok{"capping"}\NormalTok{)}
\NormalTok{wineDataset}\SpecialCharTok{$}\NormalTok{total.sulfur.dioxide }\OtherTok{\textless{}{-}} \FunctionTok{imputate\_outlier}\NormalTok{(wineDataset, total.sulfur.dioxide, }\AttributeTok{method =} \StringTok{"capping"}\NormalTok{)}

\NormalTok{wineDataset}\SpecialCharTok{$}\NormalTok{density }\OtherTok{\textless{}{-}} \FunctionTok{imputate\_outlier}\NormalTok{(wineDataset, density, }\AttributeTok{method =} \StringTok{"capping"}\NormalTok{)}
\NormalTok{wineDataset}\SpecialCharTok{$}\NormalTok{sulphates }\OtherTok{\textless{}{-}} \FunctionTok{imputate\_outlier}\NormalTok{(wineDataset, sulphates, }\AttributeTok{method =} \StringTok{"capping"}\NormalTok{)}
\end{Highlighting}
\end{Shaded}

\textbf{Comment:} Note that the default capping value of the
\texttt{imputate\_outlier} function is 95\%, therefore we don't need to
specify the capping value here.

\hypertarget{b.-inspect-the-correlation-between-the-variables.-is-there-any-collinearity-between-the-features-in-the-dataset-if-yes-provide-a-possible-solution.}{%
\subparagraph{1.b. Inspect the correlation between the variables. Is
there any collinearity between the features in the dataset? If yes,
provide a possible
solution.}\label{b.-inspect-the-correlation-between-the-variables.-is-there-any-collinearity-between-the-features-in-the-dataset-if-yes-provide-a-possible-solution.}}

\begin{Shaded}
\begin{Highlighting}[]
\FunctionTok{library}\NormalTok{(}\StringTok{"corrplot"}\NormalTok{)}
\end{Highlighting}
\end{Shaded}

\begin{verbatim}
## corrplot 0.92 loaded
\end{verbatim}

\begin{Shaded}
\begin{Highlighting}[]
\NormalTok{correlations }\OtherTok{\textless{}{-}} \FunctionTok{cor}\NormalTok{(wineDataset)}

\FunctionTok{corrplot}\NormalTok{(correlations, }\AttributeTok{type =} \StringTok{"upper"}\NormalTok{, }\AttributeTok{order =} \StringTok{"hclust"}\NormalTok{, }
         \AttributeTok{tl.col =} \StringTok{"black"}\NormalTok{, }\AttributeTok{tl.srt =} \DecValTok{45}\NormalTok{)}
\end{Highlighting}
\end{Shaded}

\includegraphics{Yenal_Arora_MATH-60604A_Final_Project_Solutions_files/figure-latex/unnamed-chunk-16-1.pdf}

\begin{Shaded}
\begin{Highlighting}[]
\FunctionTok{corrplot}\NormalTok{(correlations, }\AttributeTok{type =} \StringTok{"upper"}\NormalTok{,  }\AttributeTok{order =} \StringTok{"hclust"}\NormalTok{, }
         \AttributeTok{tl.col =} \StringTok{"black"}\NormalTok{, }\AttributeTok{tl.srt =} \DecValTok{45}\NormalTok{, }\AttributeTok{method=}\StringTok{"number"}\NormalTok{, }\AttributeTok{number.cex=}\FloatTok{0.60}\NormalTok{)}
\end{Highlighting}
\end{Shaded}

\includegraphics{Yenal_Arora_MATH-60604A_Final_Project_Solutions_files/figure-latex/unnamed-chunk-16-2.pdf}

\textbf{Comment:} From the graph, we see that the pH level and the fixed
acidity are strongly negatively correlated, while the fixed acidity and
density are strongly positively correlated, with correlation
coefficients of -0.68 and 0.67, respectively. Moreover, free sulfur
dioxide and total sulfur dioxide have a correlation coefficient of 0.67,
meaning they have a strong positive correlation.

\begin{Shaded}
\begin{Highlighting}[]
\FunctionTok{library}\NormalTok{(car)}
\end{Highlighting}
\end{Shaded}

\begin{verbatim}
## Loading required package: carData
\end{verbatim}

\begin{verbatim}
## 
## Attaching package: 'car'
\end{verbatim}

\begin{verbatim}
## The following object is masked from 'package:dplyr':
## 
##     recode
\end{verbatim}

\begin{Shaded}
\begin{Highlighting}[]
\NormalTok{mod.col1}\OtherTok{\textless{}{-}}\FunctionTok{lm}\NormalTok{(quality}\SpecialCharTok{\textasciitilde{}}\NormalTok{fixed.acidity}\SpecialCharTok{+}\NormalTok{volatile.acidity}\SpecialCharTok{+}\NormalTok{citric.acid}\SpecialCharTok{+}\NormalTok{residual.sugar}\SpecialCharTok{+}\NormalTok{chlorides}\SpecialCharTok{+}\NormalTok{free.sulfur.dioxide}\SpecialCharTok{+}\NormalTok{total.sulfur.dioxide}\SpecialCharTok{+}\NormalTok{density}\SpecialCharTok{+}\NormalTok{pH}\SpecialCharTok{+}\NormalTok{sulphates}\SpecialCharTok{+}\FunctionTok{as.factor}\NormalTok{(levels.alcohol),}\AttributeTok{data=}\NormalTok{wineDataset)}
\FunctionTok{summary}\NormalTok{(mod.col1)}
\end{Highlighting}
\end{Shaded}

\begin{verbatim}
## 
## Call:
## lm(formula = quality ~ fixed.acidity + volatile.acidity + citric.acid + 
##     residual.sugar + chlorides + free.sulfur.dioxide + total.sulfur.dioxide + 
##     density + pH + sulphates + as.factor(levels.alcohol), data = wineDataset)
## 
## Residuals:
##      Min       1Q   Median       3Q      Max 
## -2.77378 -0.37303 -0.04796  0.44652  2.02834 
## 
## Coefficients:
##                              Estimate Std. Error t value Pr(>|t|)    
## (Intercept)                 6.865e+01  2.040e+01   3.365 0.000783 ***
## fixed.acidity               7.763e-02  2.484e-02   3.125 0.001808 ** 
## volatile.acidity           -1.020e+00  1.294e-01  -7.878 6.12e-15 ***
## citric.acid                -2.722e-01  1.463e-01  -1.861 0.062908 .  
## residual.sugar              3.796e-02  2.066e-02   1.837 0.066332 .  
## chlorides                  -3.347e+00  1.001e+00  -3.344 0.000847 ***
## free.sulfur.dioxide         4.344e-03  2.477e-03   1.754 0.079707 .  
## total.sulfur.dioxide       -3.362e-03  8.591e-04  -3.913 9.49e-05 ***
## density                    -6.349e+01  2.097e+01  -3.027 0.002507 ** 
## pH                         -1.865e-01  1.889e-01  -0.987 0.323612    
## sulphates                   1.479e+00  1.400e-01  10.566  < 2e-16 ***
## as.factor(levels.alcohol)2  3.238e-01  4.818e-02   6.721 2.50e-11 ***
## as.factor(levels.alcohol)3  7.259e-01  8.148e-02   8.909  < 2e-16 ***
## ---
## Signif. codes:  0 '***' 0.001 '**' 0.01 '*' 0.05 '.' 0.1 ' ' 1
## 
## Residual standard error: 0.6499 on 1586 degrees of freedom
## Multiple R-squared:  0.3573, Adjusted R-squared:  0.3525 
## F-statistic: 73.48 on 12 and 1586 DF,  p-value: < 2.2e-16
\end{verbatim}

\begin{Shaded}
\begin{Highlighting}[]
\FunctionTok{vif}\NormalTok{(mod.col1)}
\end{Highlighting}
\end{Shaded}

\begin{verbatim}
##                               GVIF Df GVIF^(1/(2*Df))
## fixed.acidity             6.102908  1        2.470407
## volatile.acidity          1.813828  1        1.346784
## citric.acid               3.049953  1        1.746411
## residual.sugar            1.500429  1        1.224920
## chlorides                 1.308109  1        1.143726
## free.sulfur.dioxide       2.107918  1        1.451867
## total.sulfur.dioxide      2.440603  1        1.562243
## density                   4.974059  1        2.230260
## pH                        2.799577  1        1.673194
## sulphates                 1.272692  1        1.128137
## as.factor(levels.alcohol) 2.494696  2        1.256766
\end{verbatim}

\textbf{Comment:} From the VIF table, we observe that fixed acidity has
the highest VIF of 6.10, followed by density with a VIF of 4.97. We will
therefore proceed with a model ``without'' the fixed acidity variable
and see if the VIF values decrease.

\begin{Shaded}
\begin{Highlighting}[]
\NormalTok{mod.col2}\OtherTok{\textless{}{-}}\FunctionTok{lm}\NormalTok{(quality}\SpecialCharTok{\textasciitilde{}}\NormalTok{volatile.acidity}\SpecialCharTok{+}\NormalTok{citric.acid}\SpecialCharTok{+}\NormalTok{residual.sugar}\SpecialCharTok{+}\NormalTok{chlorides}\SpecialCharTok{+}\NormalTok{free.sulfur.dioxide}\SpecialCharTok{+}\NormalTok{total.sulfur.dioxide}\SpecialCharTok{+}\NormalTok{density}\SpecialCharTok{+}\NormalTok{pH}\SpecialCharTok{+}\NormalTok{sulphates}\SpecialCharTok{+}\FunctionTok{as.factor}\NormalTok{(levels.alcohol),}\AttributeTok{data=}\NormalTok{wineDataset)}
\FunctionTok{summary}\NormalTok{(mod.col2)}
\end{Highlighting}
\end{Shaded}

\begin{verbatim}
## 
## Call:
## lm(formula = quality ~ volatile.acidity + citric.acid + residual.sugar + 
##     chlorides + free.sulfur.dioxide + total.sulfur.dioxide + 
##     density + pH + sulphates + as.factor(levels.alcohol), data = wineDataset)
## 
## Residuals:
##      Min       1Q   Median       3Q      Max 
## -2.65517 -0.38253 -0.04588  0.44502  2.05982 
## 
## Coefficients:
##                              Estimate Std. Error t value Pr(>|t|)    
## (Intercept)                 2.343e+01  1.442e+01   1.625 0.104387    
## volatile.acidity           -1.002e+00  1.297e-01  -7.729 1.91e-14 ***
## citric.acid                -1.210e-01  1.384e-01  -0.874 0.382017    
## residual.sugar              1.955e-02  1.986e-02   0.984 0.325044    
## chlorides                  -3.783e+00  9.941e-01  -3.806 0.000147 ***
## free.sulfur.dioxide         5.117e-03  2.472e-03   2.070 0.038586 *  
## total.sulfur.dioxide       -4.107e-03  8.276e-04  -4.963 7.69e-07 ***
## density                    -1.614e+01  1.454e+01  -1.110 0.267209    
## pH                         -5.738e-01  1.429e-01  -4.016 6.21e-05 ***
## sulphates                   1.437e+00  1.397e-01  10.281  < 2e-16 ***
## as.factor(levels.alcohol)2  3.771e-01  4.519e-02   8.344  < 2e-16 ***
## as.factor(levels.alcohol)3  8.165e-01  7.635e-02  10.693  < 2e-16 ***
## ---
## Signif. codes:  0 '***' 0.001 '**' 0.01 '*' 0.05 '.' 0.1 ' ' 1
## 
## Residual standard error: 0.6516 on 1587 degrees of freedom
## Multiple R-squared:  0.3534, Adjusted R-squared:  0.3489 
## F-statistic: 78.84 on 11 and 1587 DF,  p-value: < 2.2e-16
\end{verbatim}

\begin{Shaded}
\begin{Highlighting}[]
\FunctionTok{vif}\NormalTok{(mod.col2)}
\end{Highlighting}
\end{Shaded}

\begin{verbatim}
##                               GVIF Df GVIF^(1/(2*Df))
## volatile.acidity          1.810440  1        1.345526
## citric.acid               2.716333  1        1.648130
## residual.sugar            1.378407  1        1.174056
## chlorides                 1.282706  1        1.132566
## free.sulfur.dioxide       2.086896  1        1.444609
## total.sulfur.dioxide      2.252512  1        1.500837
## density                   2.378154  1        1.542126
## pH                        1.593939  1        1.262513
## sulphates                 1.260527  1        1.122732
## as.factor(levels.alcohol) 2.088804  2        1.202194
\end{verbatim}

\textbf{Comment:} We see that when the fixed acidity variable is
excluded from the model, all the VIF values are below a reasonable
threshold of 3. Based on the VIF values, we remove fixed acidity.

\hypertarget{c.-inspect-the-interaction-between-alcohol-level-and-density-on-the-quality-of-the-wine.-does-there-seem-to-be-an-interaction-between-the-two-variables-comment.}{%
\subparagraph{1.c. Inspect the interaction between alcohol level and
density on the quality of the wine. Does there seem to be an interaction
between the two variables?
Comment.}\label{c.-inspect-the-interaction-between-alcohol-level-and-density-on-the-quality-of-the-wine.-does-there-seem-to-be-an-interaction-between-the-two-variables-comment.}}

\includegraphics{Yenal_Arora_MATH-60604A_Final_Project_Solutions_files/figure-latex/unnamed-chunk-19-1.pdf}

\textbf{Comment:} We see that the slopes for different levels of alcohol
are not the same, which could be a sign of interaction between alcohol
and density. More specifically, line 1 and line 3 are not parallel,
i.e., the effect of the level of alcohol on quality changes as a
function of density. We see that the higher the value of density, the
smaller the difference in quality of wine between low and high levels of
alcohol. Although this interaction looks weak.

\hypertarget{d.-model-the-quality-of-wine-in-terms-of-all-of-the-explanatory-variables-including-an-interaction-between-alcohol-and-density.-provide-the-model-summary-results.}{%
\subparagraph{1.d. Model the quality of wine in terms of all of the
explanatory variables including an interaction between alcohol and
density. Provide the model summary
results.}\label{d.-model-the-quality-of-wine-in-terms-of-all-of-the-explanatory-variables-including-an-interaction-between-alcohol-and-density.-provide-the-model-summary-results.}}

The linear regression is of the form:

\[
\begin{align}
Y = &\beta_{0} + \beta_{1}volatile.acidity + \beta_{2}citric.acid + \beta_{3}residual.sugar + \beta_{4}chlorides + \beta_{5}free.sulfur.dioxide + \\ 
&\beta_{6}total.sulfur.dioxide + \beta_{7}density + \beta_{8}pH + \beta_{9}sulphates + \beta_{10}medium.alcohol +  \beta_{11}high.alcohol +  \\ 
& \beta_{12}(medium.alcohol*density) + \beta_{13}(high.alcohol*density) + \epsilon
\end{align}
\]

\begin{Shaded}
\begin{Highlighting}[]
\NormalTok{mod}\OtherTok{\textless{}{-}}\FunctionTok{lm}\NormalTok{(quality}\SpecialCharTok{\textasciitilde{}}\NormalTok{volatile.acidity}\SpecialCharTok{+}\NormalTok{citric.acid}\SpecialCharTok{+}\NormalTok{residual.sugar}\SpecialCharTok{+}\NormalTok{chlorides}\SpecialCharTok{+}\NormalTok{free.sulfur.dioxide}\SpecialCharTok{+}\NormalTok{total.sulfur.dioxide}\SpecialCharTok{+}\NormalTok{density}\SpecialCharTok{+}\NormalTok{pH}\SpecialCharTok{+}\NormalTok{sulphates}\SpecialCharTok{+}\FunctionTok{as.factor}\NormalTok{(levels.alcohol)}\SpecialCharTok{+}\FunctionTok{as.factor}\NormalTok{(levels.alcohol)}\SpecialCharTok{*}\NormalTok{density,}\AttributeTok{data=}\NormalTok{wineDataset)}
\FunctionTok{summary}\NormalTok{(mod)}
\end{Highlighting}
\end{Shaded}

\begin{verbatim}
## 
## Call:
## lm(formula = quality ~ volatile.acidity + citric.acid + residual.sugar + 
##     chlorides + free.sulfur.dioxide + total.sulfur.dioxide + 
##     density + pH + sulphates + as.factor(levels.alcohol) + as.factor(levels.alcohol) * 
##     density, data = wineDataset)
## 
## Residuals:
##     Min      1Q  Median      3Q     Max 
## -2.6574 -0.3801 -0.0422  0.4469  2.0606 
## 
## Coefficients:
##                                      Estimate Std. Error t value Pr(>|t|)    
## (Intercept)                         2.116e+01  1.787e+01   1.184 0.236674    
## volatile.acidity                   -1.001e+00  1.299e-01  -7.703 2.33e-14 ***
## citric.acid                        -1.208e-01  1.387e-01  -0.871 0.384070    
## residual.sugar                      1.958e-02  1.995e-02   0.981 0.326542    
## chlorides                          -3.766e+00  9.986e-01  -3.772 0.000168 ***
## free.sulfur.dioxide                 5.110e-03  2.476e-03   2.064 0.039182 *  
## total.sulfur.dioxide               -4.113e-03  8.305e-04  -4.953 8.10e-07 ***
## density                            -1.385e+01  1.801e+01  -0.769 0.441777    
## pH                                 -5.759e-01  1.434e-01  -4.017 6.17e-05 ***
## sulphates                           1.435e+00  1.400e-01  10.255  < 2e-16 ***
## as.factor(levels.alcohol)2          4.967e+00  2.330e+01   0.213 0.831242    
## as.factor(levels.alcohol)3          5.978e+00  3.485e+01   0.172 0.863837    
## density:as.factor(levels.alcohol)2 -4.604e+00  2.338e+01  -0.197 0.843891    
## density:as.factor(levels.alcohol)3 -5.183e+00  3.502e+01  -0.148 0.882383    
## ---
## Signif. codes:  0 '***' 0.001 '**' 0.01 '*' 0.05 '.' 0.1 ' ' 1
## 
## Residual standard error: 0.652 on 1585 degrees of freedom
## Multiple R-squared:  0.3534, Adjusted R-squared:  0.3481 
## F-statistic: 66.63 on 13 and 1585 DF,  p-value: < 2.2e-16
\end{verbatim}

\hypertarget{e.-based-on-the-model-in-part-d-write-an-expression-for-the-fitted-model-for-each-category-of-alcohol.-comment-on-the-value-of-r2.}{%
\subparagraph{1.e. Based on the model in part (d), write an expression
for the fitted model for each category of alcohol. Comment on the value
of
R2.}\label{e.-based-on-the-model-in-part-d-write-an-expression-for-the-fitted-model-for-each-category-of-alcohol.-comment-on-the-value-of-r2.}}

The fitted model has the form:

\[
\begin{align}
\hat{quality} = &(2.116e+01) - (1.001e+00)volatile.acidity -(1.208e-01)citric.acid + \\
& (1.958e-02)residual.sugar - (3.766e+00)chlorides + (5.110e-03)free.sulfur.dioxide - \\ 
&(4.113e-03)total.sulfur.dioxide -(1.385e+01)density -(5.759e-01)pH + \\ 
& (1.435e+00)sulphates + (4.967e+00)medium.alcohol + (5.978e+00)high.alcohol - \\
&(4.604e+00)medium.alcohol*density - (5.183e+00)high.alcohol*density
\end{align}
\]

We can split up the model based on the level of alcohol:

\textbf{When the level of alcohol is} \(low\):

\[
\begin{align}
&\hat{E}(quality|alcohol=low) = (2.116e+01) - (1.001e+00)volatile.acidity - \\
&(1.208e-01)citric.acid + (1.958e-02)residual.sugar - (3.766e+00)chlorides + \\ 
& (5.110e-03)free.sulfur.dioxide -(4.113e-03)total.sulfur.dioxide -(1.385e+01)density - \\
& (5.759e-01)pH + (1.435e+00)sulphates
\end{align}
\]

\textbf{When the level of alcohol is} \(medium\):

\[
\begin{align}
&\hat{E}(quality|alcohol=medium) =  (2.116e+01 + 4.967e+00) - (1.001e+00)volatile.acidity - \\
&(1.208e-01)citric.acid + (1.958e-02)residual.sugar - (3.766e+00)chlorides + \\ 
& (5.110e-03)free.sulfur.dioxide -(4.113e-03)total.sulfur.dioxide - \\ 
& (1.385e+01+4.604e+00) density  -(5.759e-01)pH + (1.435e+00)sulphates \\ \\
& = 2.613e+01 - (1.001e+00)volatile.acidity -(1.208e-01)citric.acid + \\ 
& (1.958e-02)residual.sugar - (3.766e+00)chlorides (5.110e-03)free.sulfur.dioxide - \\ 
&(4.113e-03)total.sulfur.dioxide - 18.454 density  -(5.759e-01)pH + (1.435e+00)sulphates
\end{align}
\]

\textbf{When the level of alcohol is} \(high\):

\[
\begin{align}
&\hat{E}(quality|alcohol=high) = (2.116e+01 + 5.978e+00) - (1.001e+00)volatile.acidity - \\
&(1.208e-01)citric.acid + (1.958e-02)residual.sugar - (3.766e+00)chlorides + \\ 
& (5.110e-03)free.sulfur.dioxide -(4.113e-03)total.sulfur.dioxide - \\ 
& (1.385e+01+ 5.183e+00)density -(5.759e-01)pH + (1.435e+00)sulphates \\ \\
& = 2.714e+01 - (1.001e+00)volatile.acidity -(1.208e-01)citric.acid + \\
& (1.958e-02)residual.sugar - (3.766e+00)chlorides + (5.110e-03)free.sulfur.dioxide - \\ 
& (4.113e-03)total.sulfur.dioxide - 19.033density -(5.759e-01)pH + (1.435e+00)sulphates 
\end{align}
\]

\(R^2\) is 0.3534, meaning that all the explanatory variables in the
model and the interaction variables can explain 36.15\% of the
variability in the response variable (quality).

Adjusted \(R^2\) = 0.3481.

\hypertarget{f.-based-on-the-model-in-part-d-interpret-the-regression-coefficient-associated-with-residual-sugar-and-the-main-effect-of-density.}{%
\subparagraph{1.f. Based on the model in part (d), interpret the
regression coefficient associated with residual sugar, and the main
effect of
density.}\label{f.-based-on-the-model-in-part-d-interpret-the-regression-coefficient-associated-with-residual-sugar-and-the-main-effect-of-density.}}

\textbf{Interpreting the regression coefficient associated with residual
sugar (}\(\beta_3\)):

\[
\begin{align}
&\beta_3 = E(quality|residual sugar = x + 1) -  \\
&E(quality|residual sugar = x)
\end{align}
\]

On average, one unit increase in residual sugar parameter causes quality
of wine to increase by 1.958e-02, holding all other variables constant.

\textbf{Interpreting the regression coefficient associated with the main
effect of density.(}\(\beta_7\)):

\[
\begin{align}
&\beta_7 = E(quality|density = x + 1, alcohol=low) -  \\
&E(quality|density = x,new,alcohol=low)
\end{align}
\]

On average , one unit increase in density causes quality of wine with
low alcohol to change by -(1.385e+01) holding all other variables
constant. In other words, on average, one unit increase in density
causes quality of wine to decrease by 1.385e+01, when alcohol = low,
holding all other variables constant.

\hypertarget{g.-based-on-the-model-in-part-d-does-the-effect-of-density-on-the-quality-depend-on-the-level-of-alcohol-categorized-as-lowmediumhigh-justify-your-answer.}{%
\subparagraph{1.g. Based on the model in part (d), does the effect of
density on the quality depend on the level of alcohol? (categorized as
low/medium/high)? Justify your
answer.}\label{g.-based-on-the-model-in-part-d-does-the-effect-of-density-on-the-quality-depend-on-the-level-of-alcohol-categorized-as-lowmediumhigh-justify-your-answer.}}

Here we are interested in testing the hypotheses:

\(H_0: \beta_{12} = \beta_{13} =0\) vs.~\(H_1:\) at least one of them
\(\neq 0\)

To test the hypothesis, we proceed to build a new model called mod2,
which doesn't contain the interaction variables
\(medium.alcohol*density\) and \(high.alcohol*density\).

\begin{Shaded}
\begin{Highlighting}[]
\NormalTok{mod2 }\OtherTok{\textless{}{-}} \FunctionTok{lm}\NormalTok{(quality}\SpecialCharTok{\textasciitilde{}}\NormalTok{volatile.acidity}\SpecialCharTok{+}\NormalTok{citric.acid}\SpecialCharTok{+}\NormalTok{residual.sugar}\SpecialCharTok{+}\NormalTok{chlorides}\SpecialCharTok{+}\NormalTok{free.sulfur.dioxide}\SpecialCharTok{+}\NormalTok{total.sulfur.dioxide}\SpecialCharTok{+}\NormalTok{density}\SpecialCharTok{+}\NormalTok{pH}\SpecialCharTok{+}\NormalTok{sulphates}\SpecialCharTok{+}\FunctionTok{as.factor}\NormalTok{(levels.alcohol),}\AttributeTok{data=}\NormalTok{wineDataset)}
\end{Highlighting}
\end{Shaded}

We then compare mod and mod2 using anova function.

\begin{Shaded}
\begin{Highlighting}[]
\FunctionTok{anova}\NormalTok{(mod, mod2)}
\end{Highlighting}
\end{Shaded}

\begin{verbatim}
## Analysis of Variance Table
## 
## Model 1: quality ~ volatile.acidity + citric.acid + residual.sugar + chlorides + 
##     free.sulfur.dioxide + total.sulfur.dioxide + density + pH + 
##     sulphates + as.factor(levels.alcohol) + as.factor(levels.alcohol) * 
##     density
## Model 2: quality ~ volatile.acidity + citric.acid + residual.sugar + chlorides + 
##     free.sulfur.dioxide + total.sulfur.dioxide + density + pH + 
##     sulphates + as.factor(levels.alcohol)
##   Res.Df    RSS Df Sum of Sq      F Pr(>F)
## 1   1585 673.89                           
## 2   1587 673.91 -2 -0.020233 0.0238 0.9765
\end{verbatim}

This leads to a test statistic value (F-test) of 0.0238 and a p-value of
0.97. For the reasonable significance level of \(\alpha=0.05\), we fail
to reject \(H_0\) and conclude that the interaction between alcohol and
density is not significant, that is, the effect of density on the
quality of wine \textbf{does not} depend on the level of alcohol, and
vice-versa, keeping all other variables fixed.

\hypertarget{h.-test-whether-the-alcohol-variable-is-globally-significant-in-the-model-using-an-f-test.-justify-your-answer.-use-ux1d6fc-1.}{%
\subparagraph{1.h. Test whether the alcohol variable is globally
significant in the model, using an F-test. Justify your answer. Use 𝛼 =
1\%.}\label{h.-test-whether-the-alcohol-variable-is-globally-significant-in-the-model-using-an-f-test.-justify-your-answer.-use-ux1d6fc-1.}}

Here we are interested in testing the hypotheses:

\(H_0: \beta_{10} = \beta_{11} = \beta_{12} = \beta_{13} =0\)
vs.~\(H_1:\) at least one of them \(\neq 0\)

To test this hypothesis, we build a new model mod3, which excludes the
variables \(medium.alcohol\), \(high.alcohol\),
\(medium.alcohol*density\), and \(high.alcohol*density\).

\begin{Shaded}
\begin{Highlighting}[]
\NormalTok{mod3 }\OtherTok{\textless{}{-}} \FunctionTok{lm}\NormalTok{(quality}\SpecialCharTok{\textasciitilde{}}\NormalTok{volatile.acidity}\SpecialCharTok{+}\NormalTok{citric.acid}\SpecialCharTok{+}\NormalTok{residual.sugar}\SpecialCharTok{+}\NormalTok{chlorides}\SpecialCharTok{+}\NormalTok{free.sulfur.dioxide}\SpecialCharTok{+}\NormalTok{total.sulfur.dioxide}\SpecialCharTok{+}\NormalTok{density}\SpecialCharTok{+}\NormalTok{pH}\SpecialCharTok{+}\NormalTok{sulphates,}\AttributeTok{data=}\NormalTok{wineDataset)}
\end{Highlighting}
\end{Shaded}

We then compare mod and mod3 using anova function.

\begin{Shaded}
\begin{Highlighting}[]
\FunctionTok{anova}\NormalTok{(mod, mod3)}
\end{Highlighting}
\end{Shaded}

\begin{verbatim}
## Analysis of Variance Table
## 
## Model 1: quality ~ volatile.acidity + citric.acid + residual.sugar + chlorides + 
##     free.sulfur.dioxide + total.sulfur.dioxide + density + pH + 
##     sulphates + as.factor(levels.alcohol) + as.factor(levels.alcohol) * 
##     density
## Model 2: quality ~ volatile.acidity + citric.acid + residual.sugar + chlorides + 
##     free.sulfur.dioxide + total.sulfur.dioxide + density + pH + 
##     sulphates
##   Res.Df    RSS Df Sum of Sq      F    Pr(>F)    
## 1   1585 673.89                                  
## 2   1589 728.12 -4   -54.228 31.886 < 2.2e-16 ***
## ---
## Signif. codes:  0 '***' 0.001 '**' 0.01 '*' 0.05 '.' 0.1 ' ' 1
\end{verbatim}

This leads to a test statistic value (F-test) of 31.886 and a p-value of
2.2e-16. For the significance level of \(\alpha=0.01\), we reject
\(H_0\) and conclude that the variable alcohol is globally significant.

\hypertarget{i-based-on-the-model-in-part-d-is-there-a-significant-difference-in-the-effect-of-density-on-the-quality-of-wine-with-a-high-level-of-alcohol-in-comparison-to-wine-with-medium-level-of-alcohol-justify-your-answer.}{%
\subparagraph{1.i Based on the model in part (d), is there a significant
difference in the effect of density on the quality of wine with a high
level of alcohol in comparison to wine with medium level of alcohol?
Justify your
answer.}\label{i-based-on-the-model-in-part-d-is-there-a-significant-difference-in-the-effect-of-density-on-the-quality-of-wine-with-a-high-level-of-alcohol-in-comparison-to-wine-with-medium-level-of-alcohol-justify-your-answer.}}

In order to compare the high level of alcohol with medium level of
alcohol, we need to relevel the \(alcohol\) variable and build the model
again with the re-leveled version of \(alcohol\).

\begin{Shaded}
\begin{Highlighting}[]
\NormalTok{wineDataset}\SpecialCharTok{$}\NormalTok{levels.alcohol }\OtherTok{\textless{}{-}}\FunctionTok{relevel}\NormalTok{(}\FunctionTok{as.factor}\NormalTok{(wineDataset}\SpecialCharTok{$}\NormalTok{levels.alcohol),}\DecValTok{2}\NormalTok{)}

\FunctionTok{levels}\NormalTok{(}\FunctionTok{as.factor}\NormalTok{(wineDataset}\SpecialCharTok{$}\NormalTok{levels.alcohol))}
\end{Highlighting}
\end{Shaded}

\begin{verbatim}
## [1] "2" "1" "3"
\end{verbatim}

Now that \(medium.alcohol\) is the reference level, we can construct the
model again.

\begin{Shaded}
\begin{Highlighting}[]
\NormalTok{mod4}\OtherTok{\textless{}{-}}\FunctionTok{lm}\NormalTok{(quality}\SpecialCharTok{\textasciitilde{}}\NormalTok{volatile.acidity}\SpecialCharTok{+}\NormalTok{citric.acid}\SpecialCharTok{+}\NormalTok{residual.sugar}\SpecialCharTok{+}\NormalTok{chlorides}\SpecialCharTok{+}\NormalTok{free.sulfur.dioxide}\SpecialCharTok{+}\NormalTok{total.sulfur.dioxide}\SpecialCharTok{+}\NormalTok{density}\SpecialCharTok{+}\NormalTok{pH}\SpecialCharTok{+}\NormalTok{sulphates}\SpecialCharTok{+}\FunctionTok{as.factor}\NormalTok{(levels.alcohol)}\SpecialCharTok{+}\FunctionTok{as.factor}\NormalTok{(levels.alcohol)}\SpecialCharTok{*}\NormalTok{density,}\AttributeTok{data=}\NormalTok{wineDataset)}
\FunctionTok{summary}\NormalTok{(mod4)}
\end{Highlighting}
\end{Shaded}

\begin{verbatim}
## 
## Call:
## lm(formula = quality ~ volatile.acidity + citric.acid + residual.sugar + 
##     chlorides + free.sulfur.dioxide + total.sulfur.dioxide + 
##     density + pH + sulphates + as.factor(levels.alcohol) + as.factor(levels.alcohol) * 
##     density, data = wineDataset)
## 
## Residuals:
##     Min      1Q  Median      3Q     Max 
## -2.6574 -0.3801 -0.0422  0.4469  2.0606 
## 
## Coefficients:
##                                      Estimate Std. Error t value Pr(>|t|)    
## (Intercept)                         2.613e+01  1.976e+01   1.322 0.186300    
## volatile.acidity                   -1.001e+00  1.299e-01  -7.703 2.33e-14 ***
## citric.acid                        -1.208e-01  1.387e-01  -0.871 0.384070    
## residual.sugar                      1.958e-02  1.995e-02   0.981 0.326542    
## chlorides                          -3.766e+00  9.986e-01  -3.772 0.000168 ***
## free.sulfur.dioxide                 5.110e-03  2.476e-03   2.064 0.039182 *  
## total.sulfur.dioxide               -4.113e-03  8.305e-04  -4.953 8.10e-07 ***
## density                            -1.846e+01  1.987e+01  -0.929 0.353128    
## pH                                 -5.759e-01  1.434e-01  -4.017 6.17e-05 ***
## sulphates                           1.435e+00  1.400e-01  10.255  < 2e-16 ***
## as.factor(levels.alcohol)1         -4.967e+00  2.330e+01  -0.213 0.831242    
## as.factor(levels.alcohol)3          1.011e+00  3.570e+01   0.028 0.977406    
## density:as.factor(levels.alcohol)1  4.604e+00  2.338e+01   0.197 0.843891    
## density:as.factor(levels.alcohol)3 -5.784e-01  3.589e+01  -0.016 0.987143    
## ---
## Signif. codes:  0 '***' 0.001 '**' 0.01 '*' 0.05 '.' 0.1 ' ' 1
## 
## Residual standard error: 0.652 on 1585 degrees of freedom
## Multiple R-squared:  0.3534, Adjusted R-squared:  0.3481 
## F-statistic: 66.63 on 13 and 1585 DF,  p-value: < 2.2e-16
\end{verbatim}

We are interested in testing the hypothesis:

\(H_0: \beta_{13} =0\) vs.~\(H_1: \beta_{13} \neq 0\).

This leads to a test statistic value (t-value) of -0.016 and a p-value
of 0.987143. For the reasonable significance level of \(\alpha=0.05\),
we fail to reject \(H_0\) and conclude that the there \textbf{is not} a
significant difference in the effect of density on the quality of wine
with a high level of alcohol in comparison to wine with medium level of
alcohol.

\hypertarget{j.-carry-out-a-residual-analysis.-comment-on-the-results.}{%
\subparagraph{1.j. Carry out a residual analysis. Comment on the
results.}\label{j.-carry-out-a-residual-analysis.-comment-on-the-results.}}

In order to asses if the model is well specified and that the t-tests
are valid, we make sure that the assumption of normality is met. To do
that, we will looking at the residuals.

\begin{verbatim}
##   fixed.acidity volatile.acidity citric.acid residual.sugar chlorides
## 1           7.4             0.70        0.00            1.9     0.076
## 2           7.8             0.88        0.00            2.6     0.098
## 3           7.8             0.76        0.04            2.3     0.092
## 4          11.2             0.28        0.56            1.9     0.075
## 5           7.4             0.70        0.00            1.9     0.076
## 6           7.4             0.66        0.00            1.8     0.075
##   free.sulfur.dioxide total.sulfur.dioxide density   pH sulphates quality
## 1                  11                   34  0.9978 3.51      0.56       5
## 2                  25                   67  0.9968 3.20      0.68       5
## 3                  15                   54  0.9970 3.26      0.65       5
## 4                  17                   60  0.9980 3.16      0.58       6
## 5                  11                   34  0.9978 3.51      0.56       5
## 6                  13                   40  0.9978 3.51      0.56       5
##   levels.alcohol   fitted      resid
## 1              1 5.086284 -0.1325670
## 2              1 5.137421 -0.2114159
## 3              1 5.191389 -0.2939047
## 4              1 5.593930  0.6240677
## 5              1 5.086284 -0.1325670
## 6              1 5.113659 -0.1746367
\end{verbatim}

\begin{verbatim}
## Warning: The dot-dot notation (`..density..`) was deprecated in ggplot2 3.4.0.
## i Please use `after_stat(density)` instead.
\end{verbatim}

\includegraphics{Yenal_Arora_MATH-60604A_Final_Project_Solutions_files/figure-latex/unnamed-chunk-31-1.pdf}

\textbf{Comment:} Here it seems that the residuals are normally
distributed due to the bell shape curve of the graph. However, we see
that the residuals do not have exactly a mean of 0, which could lead us
to think that the residuals are not fully normal. Let's investigate this
comment more in details with other residuals graph.

\includegraphics{Yenal_Arora_MATH-60604A_Final_Project_Solutions_files/figure-latex/unnamed-chunk-32-1.pdf}

\textbf{Comment:} The residuals fall on average on the line, although we
can clearly see a deviation of the residuals in the quantile range (0,
-2). As most of the observation are not contained in that range, we can
assume that the residuals follow the assumption of normality.

\begin{verbatim}
## `geom_smooth()` using method = 'gam' and formula = 'y ~ s(x, bs = "cs")'
\end{verbatim}

\includegraphics{Yenal_Arora_MATH-60604A_Final_Project_Solutions_files/figure-latex/unnamed-chunk-33-1.pdf}

\textbf{Comment:} When looking at the fitted residuals of volatile
acidity on the response variable, we see that the assumptions of
normality is being respected. The residuals do not show any form of
heteroscedasticity (tunnel shape). The residuals seems to stay around
their mean.

\begin{verbatim}
## `geom_smooth()` using method = 'gam' and formula = 'y ~ s(x, bs = "cs")'
\end{verbatim}

\includegraphics{Yenal_Arora_MATH-60604A_Final_Project_Solutions_files/figure-latex/unnamed-chunk-34-1.pdf}

\textbf{Comment:} Here I can say the same thing as the previous graph,
the residuals look pretty normal and constant variance around the mean.

\begin{verbatim}
## `geom_smooth()` using method = 'gam' and formula = 'y ~ s(x, bs = "cs")'
\end{verbatim}

\includegraphics{Yenal_Arora_MATH-60604A_Final_Project_Solutions_files/figure-latex/unnamed-chunk-35-1.pdf}

\textbf{Comment:} In terms of residual sugar, I assume that the
assumption of normality is still being respected. Most of the residuals
are in the same range of value, even though we see extreme value of
residuals, they count for very few observations.

\includegraphics{Yenal_Arora_MATH-60604A_Final_Project_Solutions_files/figure-latex/unnamed-chunk-36-1.pdf}

\textbf{Comment:} The boxplots seem okay and suggest no significant
abnormalities, we could argue that the residuals for group = 1 are not
completely following the assumptions of normality, but we believe that
it is not significant enough to violate the assumptions of normality
across the group (because the mean residuals value is still very close
to zero). In short, there are no huge differences in the mean and
variances across all the levels, even though the residuals of levels of
alcohol for group = 1 doesn't have a mean residuals of 0.

\begin{verbatim}
## `geom_smooth()` using method = 'gam' and formula = 'y ~ s(x, bs = "cs")'
\end{verbatim}

\includegraphics{Yenal_Arora_MATH-60604A_Final_Project_Solutions_files/figure-latex/unnamed-chunk-37-1.pdf}

\textbf{Comment:} Finally, The residual graph concerning the fitted
values is very well fitted and looks linear.

\textbf{Conclusion:} In our residual analysis, we plotted a histogram of
residuals, a qqplot, residuals vs.~covariates, and residuals vs.~fitted
values graphs. Overall, the histogram and qqplot showed that normality
assumption of residuals is satisfied. Furthermore, residuals
vs.~covariates, and residuals vs.~fitted values graphs showed that the
relationships are indeed linear and that the assumption of constant
variance (that is variance of residuals is constant for all values of
independent variables) seems reasonable here. Therefore, the model seems
to be well specified.

\hypertarget{question-2}{%
\subsection{Question 2}\label{question-2}}

\begin{Shaded}
\begin{Highlighting}[]
\CommentTok{\# setwd("/Users/siddhantarora/Desktop/HEC/Fall 2022/Stats Modelling/Final Project")}
\CommentTok{\# df\_bank = read.csv("/Users/siddhantarora/Desktop/HEC/Fall 2022/Stats Modelling/Final Project/bank.csv",}
\CommentTok{\#                   sep = ";", header = TRUE)}

\NormalTok{df\_bank }\OtherTok{=} \FunctionTok{read.csv}\NormalTok{(}\StringTok{"/Users/philippebeliveau/Downloads/bank.csv"}\NormalTok{,}
                   \AttributeTok{sep =} \StringTok{";"}\NormalTok{, }\AttributeTok{header =} \ConstantTok{TRUE}\NormalTok{)}
\end{Highlighting}
\end{Shaded}

\textbf{Part a)}

\textbf{Perform a comprehensive exploratory data analysis (EDA) to
obtain a better understanding of the dataset and perform necessary
manipulations, if needed. Comment.}

\begin{Shaded}
\begin{Highlighting}[]
\FunctionTok{summary}\NormalTok{(df\_bank)}
\end{Highlighting}
\end{Shaded}

\begin{verbatim}
##       age            job              marital           education        
##  Min.   :19.00   Length:4521        Length:4521        Length:4521       
##  1st Qu.:33.00   Class :character   Class :character   Class :character  
##  Median :39.00   Mode  :character   Mode  :character   Mode  :character  
##  Mean   :41.17                                                           
##  3rd Qu.:49.00                                                           
##  Max.   :87.00                                                           
##    default             balance        housing              loan          
##  Length:4521        Min.   :-3313   Length:4521        Length:4521       
##  Class :character   1st Qu.:   69   Class :character   Class :character  
##  Mode  :character   Median :  444   Mode  :character   Mode  :character  
##                     Mean   : 1423                                        
##                     3rd Qu.: 1480                                        
##                     Max.   :71188                                        
##    contact               day           month              duration   
##  Length:4521        Min.   : 1.00   Length:4521        Min.   :   4  
##  Class :character   1st Qu.: 9.00   Class :character   1st Qu.: 104  
##  Mode  :character   Median :16.00   Mode  :character   Median : 185  
##                     Mean   :15.92                      Mean   : 264  
##                     3rd Qu.:21.00                      3rd Qu.: 329  
##                     Max.   :31.00                      Max.   :3025  
##     campaign          pdays           previous         poutcome        
##  Min.   : 1.000   Min.   : -1.00   Min.   : 0.0000   Length:4521       
##  1st Qu.: 1.000   1st Qu.: -1.00   1st Qu.: 0.0000   Class :character  
##  Median : 2.000   Median : -1.00   Median : 0.0000   Mode  :character  
##  Mean   : 2.794   Mean   : 39.77   Mean   : 0.5426                     
##  3rd Qu.: 3.000   3rd Qu.: -1.00   3rd Qu.: 0.0000                     
##  Max.   :50.000   Max.   :871.00   Max.   :25.0000                     
##       y            
##  Length:4521       
##  Class :character  
##  Mode  :character  
##                    
##                    
## 
\end{verbatim}

\begin{Shaded}
\begin{Highlighting}[]
\FunctionTok{sum}\NormalTok{(}\FunctionTok{is.na}\NormalTok{(df\_bank))}
\end{Highlighting}
\end{Shaded}

\begin{verbatim}
## [1] 0
\end{verbatim}

\begin{Shaded}
\begin{Highlighting}[]
\FunctionTok{dim}\NormalTok{(df\_bank)}
\end{Highlighting}
\end{Shaded}

\begin{verbatim}
## [1] 4521   17
\end{verbatim}

\begin{Shaded}
\begin{Highlighting}[]
\FunctionTok{str}\NormalTok{(df\_bank)}
\end{Highlighting}
\end{Shaded}

\begin{verbatim}
## 'data.frame':    4521 obs. of  17 variables:
##  $ age      : int  30 33 35 30 59 35 36 39 41 43 ...
##  $ job      : chr  "unemployed" "services" "management" "management" ...
##  $ marital  : chr  "married" "married" "single" "married" ...
##  $ education: chr  "primary" "secondary" "tertiary" "tertiary" ...
##  $ default  : chr  "no" "no" "no" "no" ...
##  $ balance  : int  1787 4789 1350 1476 0 747 307 147 221 -88 ...
##  $ housing  : chr  "no" "yes" "yes" "yes" ...
##  $ loan     : chr  "no" "yes" "no" "yes" ...
##  $ contact  : chr  "cellular" "cellular" "cellular" "unknown" ...
##  $ day      : int  19 11 16 3 5 23 14 6 14 17 ...
##  $ month    : chr  "oct" "may" "apr" "jun" ...
##  $ duration : int  79 220 185 199 226 141 341 151 57 313 ...
##  $ campaign : int  1 1 1 4 1 2 1 2 2 1 ...
##  $ pdays    : int  -1 339 330 -1 -1 176 330 -1 -1 147 ...
##  $ previous : int  0 4 1 0 0 3 2 0 0 2 ...
##  $ poutcome : chr  "unknown" "failure" "failure" "unknown" ...
##  $ y        : chr  "no" "no" "no" "no" ...
\end{verbatim}

\begin{Shaded}
\begin{Highlighting}[]
\FunctionTok{head}\NormalTok{(df\_bank)}
\end{Highlighting}
\end{Shaded}

\begin{verbatim}
##   age         job marital education default balance housing loan  contact day
## 1  30  unemployed married   primary      no    1787      no   no cellular  19
## 2  33    services married secondary      no    4789     yes  yes cellular  11
## 3  35  management  single  tertiary      no    1350     yes   no cellular  16
## 4  30  management married  tertiary      no    1476     yes  yes  unknown   3
## 5  59 blue-collar married secondary      no       0     yes   no  unknown   5
## 6  35  management  single  tertiary      no     747      no   no cellular  23
##   month duration campaign pdays previous poutcome  y
## 1   oct       79        1    -1        0  unknown no
## 2   may      220        1   339        4  failure no
## 3   apr      185        1   330        1  failure no
## 4   jun      199        4    -1        0  unknown no
## 5   may      226        1    -1        0  unknown no
## 6   feb      141        2   176        3  failure no
\end{verbatim}

\begin{Shaded}
\begin{Highlighting}[]
\CommentTok{\#Converting character values to factors}
\NormalTok{col\_list }\OtherTok{\textless{}{-}} \FunctionTok{c}\NormalTok{(}\StringTok{"job"}\NormalTok{, }\StringTok{"marital"}\NormalTok{,}\StringTok{"education"}\NormalTok{, }\StringTok{"default"}\NormalTok{, }\StringTok{"housing"}\NormalTok{,}\StringTok{"loan"}\NormalTok{, }
              \StringTok{"contact"}\NormalTok{,}\StringTok{"month"}\NormalTok{, }\StringTok{"poutcome"}\NormalTok{,}\StringTok{"y"}\NormalTok{)}
\ControlFlowTok{for}\NormalTok{ (col }\ControlFlowTok{in}\NormalTok{ col\_list) \{}
\NormalTok{  df\_bank[[col]] }\OtherTok{\textless{}{-}} \FunctionTok{as.factor}\NormalTok{(df\_bank[[col]])}
\NormalTok{\}}
\end{Highlighting}
\end{Shaded}

\textbf{Correlations}

\begin{Shaded}
\begin{Highlighting}[]
\NormalTok{num\_list }\OtherTok{\textless{}{-}}\NormalTok{ df\_bank[,}\FunctionTok{c}\NormalTok{(}\DecValTok{1}\NormalTok{,}\DecValTok{6}\NormalTok{,}\DecValTok{10}\NormalTok{,}\DecValTok{12}\NormalTok{,}\DecValTok{13}\NormalTok{,}\DecValTok{14}\NormalTok{,}\DecValTok{15}\NormalTok{)]}
\FunctionTok{library}\NormalTok{(corrplot)}
\NormalTok{C }\OtherTok{\textless{}{-}} \FunctionTok{cor}\NormalTok{(num\_list)}
\FunctionTok{corrplot}\NormalTok{(C, }\AttributeTok{method=}\StringTok{"circle"}\NormalTok{)}
\end{Highlighting}
\end{Shaded}

\includegraphics{Yenal_Arora_MATH-60604A_Final_Project_Solutions_files/figure-latex/unnamed-chunk-40-1.pdf}

We can see that pdays and previous have a high correlation. This
suggests that at the time of model building, it is best to include
either one of them.

\textbf{Distribution of the variables}

\begin{Shaded}
\begin{Highlighting}[]
\CommentTok{\#Job}
\FunctionTok{library}\NormalTok{(ggplot2)}
\NormalTok{AA }\OtherTok{\textless{}{-}} \FunctionTok{ggplot}\NormalTok{(df\_bank, }\FunctionTok{aes}\NormalTok{(job))}
\NormalTok{AA }\OtherTok{\textless{}{-}}\NormalTok{ AA  }\SpecialCharTok{+} \FunctionTok{geom\_histogram}\NormalTok{(}\AttributeTok{stat=}\StringTok{"count"}\NormalTok{) }\SpecialCharTok{+} \FunctionTok{labs}\NormalTok{(}\AttributeTok{title =} \StringTok{"Job"}\NormalTok{)}\SpecialCharTok{+}
  \FunctionTok{theme}\NormalTok{(}\AttributeTok{axis.text.x=}\FunctionTok{element\_text}\NormalTok{(}\AttributeTok{angle=}\DecValTok{90}\NormalTok{,}\AttributeTok{hjust=}\DecValTok{1}\NormalTok{,}\AttributeTok{vjust=}\FloatTok{0.5}\NormalTok{))}
\end{Highlighting}
\end{Shaded}

\begin{verbatim}
## Warning in geom_histogram(stat = "count"): Ignoring unknown parameters:
## `binwidth`, `bins`, and `pad`
\end{verbatim}

\begin{Shaded}
\begin{Highlighting}[]
\NormalTok{AA}
\end{Highlighting}
\end{Shaded}

\includegraphics{Yenal_Arora_MATH-60604A_Final_Project_Solutions_files/figure-latex/unnamed-chunk-41-1.pdf}

\begin{Shaded}
\begin{Highlighting}[]
\FunctionTok{table}\NormalTok{(df\_bank}\SpecialCharTok{$}\NormalTok{job)}
\end{Highlighting}
\end{Shaded}

\begin{verbatim}
## 
##        admin.   blue-collar  entrepreneur     housemaid    management 
##           478           946           168           112           969 
##       retired self-employed      services       student    technician 
##           230           183           417            84           768 
##    unemployed       unknown 
##           128            38
\end{verbatim}

\begin{Shaded}
\begin{Highlighting}[]
\FunctionTok{round}\NormalTok{((}\FunctionTok{prop.table}\NormalTok{(}\FunctionTok{table}\NormalTok{(df\_bank}\SpecialCharTok{$}\NormalTok{job))}\SpecialCharTok{*}\DecValTok{100}\NormalTok{),}\DecValTok{1}\NormalTok{)}
\end{Highlighting}
\end{Shaded}

\begin{verbatim}
## 
##        admin.   blue-collar  entrepreneur     housemaid    management 
##          10.6          20.9           3.7           2.5          21.4 
##       retired self-employed      services       student    technician 
##           5.1           4.0           9.2           1.9          17.0 
##    unemployed       unknown 
##           2.8           0.8
\end{verbatim}

Approximately 60\% of the jobs fall into just 3 categories in our
dataset: management, blue-collar and technician, suggesting that the
variable is not that well-balanced across all levels of the job.

\begin{Shaded}
\begin{Highlighting}[]
\CommentTok{\#Marital}

\NormalTok{BB }\OtherTok{\textless{}{-}} \FunctionTok{ggplot}\NormalTok{(df\_bank, }\FunctionTok{aes}\NormalTok{(marital))}
\NormalTok{BB }\OtherTok{\textless{}{-}}\NormalTok{ BB  }\SpecialCharTok{+} \FunctionTok{geom\_histogram}\NormalTok{(}\AttributeTok{stat=}\StringTok{"count"}\NormalTok{) }\SpecialCharTok{+} \FunctionTok{labs}\NormalTok{(}\AttributeTok{title =} \StringTok{"Marital"}\NormalTok{)}\SpecialCharTok{+}
  \FunctionTok{theme}\NormalTok{(}\AttributeTok{axis.text.x=}\FunctionTok{element\_text}\NormalTok{(}\AttributeTok{angle=}\DecValTok{90}\NormalTok{,}\AttributeTok{hjust=}\DecValTok{1}\NormalTok{,}\AttributeTok{vjust=}\FloatTok{0.5}\NormalTok{))}
\end{Highlighting}
\end{Shaded}

\begin{verbatim}
## Warning in geom_histogram(stat = "count"): Ignoring unknown parameters:
## `binwidth`, `bins`, and `pad`
\end{verbatim}

\begin{Shaded}
\begin{Highlighting}[]
\NormalTok{BB}
\end{Highlighting}
\end{Shaded}

\includegraphics{Yenal_Arora_MATH-60604A_Final_Project_Solutions_files/figure-latex/unnamed-chunk-42-1.pdf}

\begin{Shaded}
\begin{Highlighting}[]
\FunctionTok{table}\NormalTok{(df\_bank}\SpecialCharTok{$}\NormalTok{marital)}
\end{Highlighting}
\end{Shaded}

\begin{verbatim}
## 
## divorced  married   single 
##      528     2797     1196
\end{verbatim}

\begin{Shaded}
\begin{Highlighting}[]
\FunctionTok{round}\NormalTok{((}\FunctionTok{prop.table}\NormalTok{(}\FunctionTok{table}\NormalTok{(df\_bank}\SpecialCharTok{$}\NormalTok{marital))}\SpecialCharTok{*}\DecValTok{100}\NormalTok{),}\DecValTok{1}\NormalTok{)}
\end{Highlighting}
\end{Shaded}

\begin{verbatim}
## 
## divorced  married   single 
##     11.7     61.9     26.5
\end{verbatim}

Majority individuals are married. This variable is relatively well
balanced if we group single with divorced.

\begin{Shaded}
\begin{Highlighting}[]
\CommentTok{\#Default}

\NormalTok{CC }\OtherTok{\textless{}{-}} \FunctionTok{ggplot}\NormalTok{(df\_bank, }\FunctionTok{aes}\NormalTok{(default))}
\NormalTok{CC }\OtherTok{\textless{}{-}}\NormalTok{ CC  }\SpecialCharTok{+} \FunctionTok{geom\_histogram}\NormalTok{(}\AttributeTok{stat=}\StringTok{"count"}\NormalTok{) }\SpecialCharTok{+} \FunctionTok{labs}\NormalTok{(}\AttributeTok{title =} \StringTok{"Default"}\NormalTok{)}\SpecialCharTok{+}
  \FunctionTok{theme}\NormalTok{(}\AttributeTok{axis.text.x=}\FunctionTok{element\_text}\NormalTok{(}\AttributeTok{angle=}\DecValTok{90}\NormalTok{,}\AttributeTok{hjust=}\DecValTok{1}\NormalTok{,}\AttributeTok{vjust=}\FloatTok{0.5}\NormalTok{))}
\end{Highlighting}
\end{Shaded}

\begin{verbatim}
## Warning in geom_histogram(stat = "count"): Ignoring unknown parameters:
## `binwidth`, `bins`, and `pad`
\end{verbatim}

\begin{Shaded}
\begin{Highlighting}[]
\NormalTok{CC}
\end{Highlighting}
\end{Shaded}

\includegraphics{Yenal_Arora_MATH-60604A_Final_Project_Solutions_files/figure-latex/unnamed-chunk-43-1.pdf}

\begin{Shaded}
\begin{Highlighting}[]
\FunctionTok{table}\NormalTok{(df\_bank}\SpecialCharTok{$}\NormalTok{default)}
\end{Highlighting}
\end{Shaded}

\begin{verbatim}
## 
##   no  yes 
## 4445   76
\end{verbatim}

\begin{Shaded}
\begin{Highlighting}[]
\FunctionTok{round}\NormalTok{((}\FunctionTok{prop.table}\NormalTok{(}\FunctionTok{table}\NormalTok{(df\_bank}\SpecialCharTok{$}\NormalTok{default))}\SpecialCharTok{*}\DecValTok{100}\NormalTok{),}\DecValTok{1}\NormalTok{)}
\end{Highlighting}
\end{Shaded}

\begin{verbatim}
## 
##   no  yes 
## 98.3  1.7
\end{verbatim}

There is only 1.7\% of the individuals who have defaulted. The
variable's predictive power would be very less, given that the
distribution is highly unbalanced.

\begin{Shaded}
\begin{Highlighting}[]
\CommentTok{\#Education}

\NormalTok{DD }\OtherTok{\textless{}{-}} \FunctionTok{ggplot}\NormalTok{(df\_bank, }\FunctionTok{aes}\NormalTok{(education))}
\NormalTok{DD }\OtherTok{\textless{}{-}}\NormalTok{ DD }\SpecialCharTok{+} \FunctionTok{geom\_histogram}\NormalTok{(}\AttributeTok{stat=}\StringTok{"count"}\NormalTok{) }\SpecialCharTok{+} \FunctionTok{labs}\NormalTok{(}\AttributeTok{title =} \StringTok{"Education"}\NormalTok{)}\SpecialCharTok{+}
  \FunctionTok{theme}\NormalTok{(}\AttributeTok{axis.text.x=}\FunctionTok{element\_text}\NormalTok{(}\AttributeTok{angle=}\DecValTok{90}\NormalTok{,}\AttributeTok{hjust=}\DecValTok{1}\NormalTok{,}\AttributeTok{vjust=}\FloatTok{0.5}\NormalTok{))}
\end{Highlighting}
\end{Shaded}

\begin{verbatim}
## Warning in geom_histogram(stat = "count"): Ignoring unknown parameters:
## `binwidth`, `bins`, and `pad`
\end{verbatim}

\begin{Shaded}
\begin{Highlighting}[]
\NormalTok{DD}
\end{Highlighting}
\end{Shaded}

\includegraphics{Yenal_Arora_MATH-60604A_Final_Project_Solutions_files/figure-latex/unnamed-chunk-44-1.pdf}

\begin{Shaded}
\begin{Highlighting}[]
\FunctionTok{table}\NormalTok{(df\_bank}\SpecialCharTok{$}\NormalTok{education)}
\end{Highlighting}
\end{Shaded}

\begin{verbatim}
## 
##   primary secondary  tertiary   unknown 
##       678      2306      1350       187
\end{verbatim}

\begin{Shaded}
\begin{Highlighting}[]
\FunctionTok{round}\NormalTok{((}\FunctionTok{prop.table}\NormalTok{(}\FunctionTok{table}\NormalTok{(df\_bank}\SpecialCharTok{$}\NormalTok{education))}\SpecialCharTok{*}\DecValTok{100}\NormalTok{),}\DecValTok{1}\NormalTok{)}
\end{Highlighting}
\end{Shaded}

\begin{verbatim}
## 
##   primary secondary  tertiary   unknown 
##      15.0      51.0      29.9       4.1
\end{verbatim}

51\% of the individuals have a secondary education. There are some
unknown observations but they are very less (4.1\%) and probably they
should not hamper the modelling results.

\begin{Shaded}
\begin{Highlighting}[]
\CommentTok{\#Loan}

\NormalTok{EE }\OtherTok{\textless{}{-}} \FunctionTok{ggplot}\NormalTok{(df\_bank, }\FunctionTok{aes}\NormalTok{(loan))}
\NormalTok{EE }\OtherTok{\textless{}{-}}\NormalTok{ EE }\SpecialCharTok{+} \FunctionTok{geom\_histogram}\NormalTok{(}\AttributeTok{stat=}\StringTok{"count"}\NormalTok{) }\SpecialCharTok{+} \FunctionTok{labs}\NormalTok{(}\AttributeTok{title =} \StringTok{"Loan"}\NormalTok{)}\SpecialCharTok{+}
  \FunctionTok{theme}\NormalTok{(}\AttributeTok{axis.text.x=}\FunctionTok{element\_text}\NormalTok{(}\AttributeTok{angle=}\DecValTok{90}\NormalTok{,}\AttributeTok{hjust=}\DecValTok{1}\NormalTok{,}\AttributeTok{vjust=}\FloatTok{0.5}\NormalTok{))}
\end{Highlighting}
\end{Shaded}

\begin{verbatim}
## Warning in geom_histogram(stat = "count"): Ignoring unknown parameters:
## `binwidth`, `bins`, and `pad`
\end{verbatim}

\begin{Shaded}
\begin{Highlighting}[]
\NormalTok{EE}
\end{Highlighting}
\end{Shaded}

\includegraphics{Yenal_Arora_MATH-60604A_Final_Project_Solutions_files/figure-latex/unnamed-chunk-45-1.pdf}

\begin{Shaded}
\begin{Highlighting}[]
\FunctionTok{table}\NormalTok{(df\_bank}\SpecialCharTok{$}\NormalTok{loan)}
\end{Highlighting}
\end{Shaded}

\begin{verbatim}
## 
##   no  yes 
## 3830  691
\end{verbatim}

\begin{Shaded}
\begin{Highlighting}[]
\FunctionTok{round}\NormalTok{((}\FunctionTok{prop.table}\NormalTok{(}\FunctionTok{table}\NormalTok{(df\_bank}\SpecialCharTok{$}\NormalTok{loan))}\SpecialCharTok{*}\DecValTok{100}\NormalTok{),}\DecValTok{1}\NormalTok{)}
\end{Highlighting}
\end{Shaded}

\begin{verbatim}
## 
##   no  yes 
## 84.7 15.3
\end{verbatim}

Approximately 85\% of the individuals did not have a personal loan.

\begin{Shaded}
\begin{Highlighting}[]
\CommentTok{\#Housing}

\NormalTok{FF }\OtherTok{\textless{}{-}} \FunctionTok{ggplot}\NormalTok{(df\_bank, }\FunctionTok{aes}\NormalTok{(housing))}
\NormalTok{FF }\OtherTok{\textless{}{-}}\NormalTok{ FF }\SpecialCharTok{+} \FunctionTok{geom\_histogram}\NormalTok{(}\AttributeTok{stat=}\StringTok{"count"}\NormalTok{) }\SpecialCharTok{+} \FunctionTok{labs}\NormalTok{(}\AttributeTok{title =} \StringTok{"Housing Loan"}\NormalTok{)}\SpecialCharTok{+}
  \FunctionTok{theme}\NormalTok{(}\AttributeTok{axis.text.x=}\FunctionTok{element\_text}\NormalTok{(}\AttributeTok{angle=}\DecValTok{90}\NormalTok{,}\AttributeTok{hjust=}\DecValTok{1}\NormalTok{,}\AttributeTok{vjust=}\FloatTok{0.5}\NormalTok{))}
\end{Highlighting}
\end{Shaded}

\begin{verbatim}
## Warning in geom_histogram(stat = "count"): Ignoring unknown parameters:
## `binwidth`, `bins`, and `pad`
\end{verbatim}

\begin{Shaded}
\begin{Highlighting}[]
\NormalTok{FF}
\end{Highlighting}
\end{Shaded}

\includegraphics{Yenal_Arora_MATH-60604A_Final_Project_Solutions_files/figure-latex/unnamed-chunk-46-1.pdf}

\begin{Shaded}
\begin{Highlighting}[]
\FunctionTok{table}\NormalTok{(df\_bank}\SpecialCharTok{$}\NormalTok{housing)}
\end{Highlighting}
\end{Shaded}

\begin{verbatim}
## 
##   no  yes 
## 1962 2559
\end{verbatim}

\begin{Shaded}
\begin{Highlighting}[]
\FunctionTok{round}\NormalTok{((}\FunctionTok{prop.table}\NormalTok{(}\FunctionTok{table}\NormalTok{(df\_bank}\SpecialCharTok{$}\NormalTok{housing))}\SpecialCharTok{*}\DecValTok{100}\NormalTok{),}\DecValTok{1}\NormalTok{)}
\end{Highlighting}
\end{Shaded}

\begin{verbatim}
## 
##   no  yes 
## 43.4 56.6
\end{verbatim}

The distribution here is very well balanced between the individuals who
have a housing loan vs.~who have no housing loan.

\begin{Shaded}
\begin{Highlighting}[]
\CommentTok{\#Contact}

\NormalTok{GG }\OtherTok{\textless{}{-}} \FunctionTok{ggplot}\NormalTok{(df\_bank, }\FunctionTok{aes}\NormalTok{(contact))}
\NormalTok{GG }\OtherTok{\textless{}{-}}\NormalTok{ GG }\SpecialCharTok{+} \FunctionTok{geom\_histogram}\NormalTok{(}\AttributeTok{stat=}\StringTok{"count"}\NormalTok{) }\SpecialCharTok{+} \FunctionTok{labs}\NormalTok{(}\AttributeTok{title =} \StringTok{"Contact"}\NormalTok{)}\SpecialCharTok{+}
  \FunctionTok{theme}\NormalTok{(}\AttributeTok{axis.text.x=}\FunctionTok{element\_text}\NormalTok{(}\AttributeTok{angle=}\DecValTok{90}\NormalTok{,}\AttributeTok{hjust=}\DecValTok{1}\NormalTok{,}\AttributeTok{vjust=}\FloatTok{0.5}\NormalTok{))}
\end{Highlighting}
\end{Shaded}

\begin{verbatim}
## Warning in geom_histogram(stat = "count"): Ignoring unknown parameters:
## `binwidth`, `bins`, and `pad`
\end{verbatim}

\begin{Shaded}
\begin{Highlighting}[]
\NormalTok{GG}
\end{Highlighting}
\end{Shaded}

\includegraphics{Yenal_Arora_MATH-60604A_Final_Project_Solutions_files/figure-latex/unnamed-chunk-47-1.pdf}

\begin{Shaded}
\begin{Highlighting}[]
\FunctionTok{table}\NormalTok{(df\_bank}\SpecialCharTok{$}\NormalTok{contact)}
\end{Highlighting}
\end{Shaded}

\begin{verbatim}
## 
##  cellular telephone   unknown 
##      2896       301      1324
\end{verbatim}

\begin{Shaded}
\begin{Highlighting}[]
\FunctionTok{round}\NormalTok{((}\FunctionTok{prop.table}\NormalTok{(}\FunctionTok{table}\NormalTok{(df\_bank}\SpecialCharTok{$}\NormalTok{contact))}\SpecialCharTok{*}\DecValTok{100}\NormalTok{),}\DecValTok{1}\NormalTok{)}
\end{Highlighting}
\end{Shaded}

\begin{verbatim}
## 
##  cellular telephone   unknown 
##      64.1       6.7      29.3
\end{verbatim}

64.1\% of the individuals were contacted via cellular means whereas only
approx. 7\% were contacted via telephone with 30\% unknown data.

\begin{Shaded}
\begin{Highlighting}[]
\CommentTok{\#Month}

\NormalTok{HH }\OtherTok{\textless{}{-}} \FunctionTok{ggplot}\NormalTok{(df\_bank, }\FunctionTok{aes}\NormalTok{(month))}
\NormalTok{HH }\OtherTok{\textless{}{-}}\NormalTok{ HH }\SpecialCharTok{+} \FunctionTok{geom\_histogram}\NormalTok{(}\AttributeTok{stat=}\StringTok{"count"}\NormalTok{) }\SpecialCharTok{+} \FunctionTok{labs}\NormalTok{(}\AttributeTok{title =} \StringTok{"Month"}\NormalTok{)}\SpecialCharTok{+}
  \FunctionTok{theme}\NormalTok{(}\AttributeTok{axis.text.x=}\FunctionTok{element\_text}\NormalTok{(}\AttributeTok{angle=}\DecValTok{90}\NormalTok{,}\AttributeTok{hjust=}\DecValTok{1}\NormalTok{,}\AttributeTok{vjust=}\FloatTok{0.5}\NormalTok{))}
\end{Highlighting}
\end{Shaded}

\begin{verbatim}
## Warning in geom_histogram(stat = "count"): Ignoring unknown parameters:
## `binwidth`, `bins`, and `pad`
\end{verbatim}

\begin{Shaded}
\begin{Highlighting}[]
\NormalTok{HH}
\end{Highlighting}
\end{Shaded}

\includegraphics{Yenal_Arora_MATH-60604A_Final_Project_Solutions_files/figure-latex/unnamed-chunk-48-1.pdf}

\begin{Shaded}
\begin{Highlighting}[]
\FunctionTok{table}\NormalTok{(df\_bank}\SpecialCharTok{$}\NormalTok{month)}
\end{Highlighting}
\end{Shaded}

\begin{verbatim}
## 
##  apr  aug  dec  feb  jan  jul  jun  mar  may  nov  oct  sep 
##  293  633   20  222  148  706  531   49 1398  389   80   52
\end{verbatim}

\begin{Shaded}
\begin{Highlighting}[]
\FunctionTok{round}\NormalTok{((}\FunctionTok{prop.table}\NormalTok{(}\FunctionTok{table}\NormalTok{(df\_bank}\SpecialCharTok{$}\NormalTok{month))}\SpecialCharTok{*}\DecValTok{100}\NormalTok{),}\DecValTok{1}\NormalTok{)}
\end{Highlighting}
\end{Shaded}

\begin{verbatim}
## 
##  apr  aug  dec  feb  jan  jul  jun  mar  may  nov  oct  sep 
##  6.5 14.0  0.4  4.9  3.3 15.6 11.7  1.1 30.9  8.6  1.8  1.2
\end{verbatim}

Majority of the individuals were contacted mostly during the summer
months with a sizable portion in the month of May (30.9\%). It might
show that contacting people during certain months can have a favorable
or unfavorable impact on the bank's marketing campaign to subscribe to
term deposits.

\begin{Shaded}
\begin{Highlighting}[]
\CommentTok{\#Poutcome}

\NormalTok{II }\OtherTok{\textless{}{-}} \FunctionTok{ggplot}\NormalTok{(df\_bank, }\FunctionTok{aes}\NormalTok{(poutcome))}
\NormalTok{II }\OtherTok{\textless{}{-}}\NormalTok{ II }\SpecialCharTok{+} \FunctionTok{geom\_histogram}\NormalTok{(}\AttributeTok{stat=}\StringTok{"count"}\NormalTok{) }\SpecialCharTok{+} \FunctionTok{labs}\NormalTok{(}\AttributeTok{title =} \StringTok{"Poutcome"}\NormalTok{)}\SpecialCharTok{+}
  \FunctionTok{theme}\NormalTok{(}\AttributeTok{axis.text.x=}\FunctionTok{element\_text}\NormalTok{(}\AttributeTok{angle=}\DecValTok{90}\NormalTok{,}\AttributeTok{hjust=}\DecValTok{1}\NormalTok{,}\AttributeTok{vjust=}\FloatTok{0.5}\NormalTok{))}
\end{Highlighting}
\end{Shaded}

\begin{verbatim}
## Warning in geom_histogram(stat = "count"): Ignoring unknown parameters:
## `binwidth`, `bins`, and `pad`
\end{verbatim}

\begin{Shaded}
\begin{Highlighting}[]
\NormalTok{II}
\end{Highlighting}
\end{Shaded}

\includegraphics{Yenal_Arora_MATH-60604A_Final_Project_Solutions_files/figure-latex/unnamed-chunk-49-1.pdf}

\begin{Shaded}
\begin{Highlighting}[]
\FunctionTok{table}\NormalTok{(df\_bank}\SpecialCharTok{$}\NormalTok{poutcome)}
\end{Highlighting}
\end{Shaded}

\begin{verbatim}
## 
## failure   other success unknown 
##     490     197     129    3705
\end{verbatim}

\begin{Shaded}
\begin{Highlighting}[]
\FunctionTok{round}\NormalTok{((}\FunctionTok{prop.table}\NormalTok{(}\FunctionTok{table}\NormalTok{(df\_bank}\SpecialCharTok{$}\NormalTok{poutcome))}\SpecialCharTok{*}\DecValTok{100}\NormalTok{),}\DecValTok{1}\NormalTok{)}
\end{Highlighting}
\end{Shaded}

\begin{verbatim}
## 
## failure   other success unknown 
##    10.8     4.4     2.9    82.0
\end{verbatim}

The value is unknown for 82\% of the individuals. We could treat these
unknowns as missing data (NA), however, these unknowns could be the
individuals which the bank has not contacted yet. Therefore, first it is
important to check the relation of this variable with our target
variable (y) to see how many individuals from this unknown category
subscribe to the term deposit.

\begin{Shaded}
\begin{Highlighting}[]
\CommentTok{\#distribution of the response variable y}

\NormalTok{JJ }\OtherTok{\textless{}{-}} \FunctionTok{ggplot}\NormalTok{(df\_bank, }\FunctionTok{aes}\NormalTok{(y))}
\NormalTok{JJ }\OtherTok{\textless{}{-}}\NormalTok{ JJ }\SpecialCharTok{+} \FunctionTok{geom\_histogram}\NormalTok{(}\AttributeTok{stat=}\StringTok{"count"}\NormalTok{) }\SpecialCharTok{+} \FunctionTok{labs}\NormalTok{(}\AttributeTok{title =} \StringTok{"Response Variable: y"}\NormalTok{)}\SpecialCharTok{+}
  \FunctionTok{theme}\NormalTok{(}\AttributeTok{axis.text.x=}\FunctionTok{element\_text}\NormalTok{(}\AttributeTok{angle=}\DecValTok{90}\NormalTok{,}\AttributeTok{hjust=}\DecValTok{1}\NormalTok{,}\AttributeTok{vjust=}\FloatTok{0.5}\NormalTok{))}
\end{Highlighting}
\end{Shaded}

\begin{verbatim}
## Warning in geom_histogram(stat = "count"): Ignoring unknown parameters:
## `binwidth`, `bins`, and `pad`
\end{verbatim}

\begin{Shaded}
\begin{Highlighting}[]
\NormalTok{JJ}
\end{Highlighting}
\end{Shaded}

\includegraphics{Yenal_Arora_MATH-60604A_Final_Project_Solutions_files/figure-latex/unnamed-chunk-50-1.pdf}

\begin{Shaded}
\begin{Highlighting}[]
\FunctionTok{table}\NormalTok{(df\_bank}\SpecialCharTok{$}\NormalTok{y)}
\end{Highlighting}
\end{Shaded}

\begin{verbatim}
## 
##   no  yes 
## 4000  521
\end{verbatim}

\begin{Shaded}
\begin{Highlighting}[]
\FunctionTok{round}\NormalTok{((}\FunctionTok{prop.table}\NormalTok{(}\FunctionTok{table}\NormalTok{(df\_bank}\SpecialCharTok{$}\NormalTok{y))}\SpecialCharTok{*}\DecValTok{100}\NormalTok{),}\DecValTok{1}\NormalTok{)}
\end{Highlighting}
\end{Shaded}

\begin{verbatim}
## 
##   no  yes 
## 88.5 11.5
\end{verbatim}

Only around 11.5\% of respondents to the current campaign have
subscribed to the bank's term deposit. This suggests that the dataset is
unbalanced.

\begin{Shaded}
\begin{Highlighting}[]
\CommentTok{\#Age}

\NormalTok{KK }\OtherTok{\textless{}{-}} \FunctionTok{ggplot}\NormalTok{(df\_bank, }\FunctionTok{aes}\NormalTok{(age))}
\NormalTok{KK }\OtherTok{\textless{}{-}}\NormalTok{ KK }\SpecialCharTok{+} \FunctionTok{geom\_histogram}\NormalTok{(}\AttributeTok{stat=}\StringTok{"count"}\NormalTok{) }\SpecialCharTok{+} \FunctionTok{labs}\NormalTok{(}\AttributeTok{title =} \StringTok{"Age"}\NormalTok{)}\SpecialCharTok{+}
  \FunctionTok{theme}\NormalTok{(}\AttributeTok{axis.text.x=}\FunctionTok{element\_text}\NormalTok{(}\AttributeTok{angle=}\DecValTok{90}\NormalTok{,}\AttributeTok{hjust=}\DecValTok{1}\NormalTok{,}\AttributeTok{vjust=}\FloatTok{0.5}\NormalTok{))}
\end{Highlighting}
\end{Shaded}

\begin{verbatim}
## Warning in geom_histogram(stat = "count"): Ignoring unknown parameters:
## `binwidth`, `bins`, and `pad`
\end{verbatim}

\begin{Shaded}
\begin{Highlighting}[]
\NormalTok{KK}
\end{Highlighting}
\end{Shaded}

\includegraphics{Yenal_Arora_MATH-60604A_Final_Project_Solutions_files/figure-latex/unnamed-chunk-51-1.pdf}

The distribution here seems to be fairly balanced where the majority
individuals are aged between 25 to 50 years old.

\begin{Shaded}
\begin{Highlighting}[]
\CommentTok{\#Balance}
\FunctionTok{library}\NormalTok{(ggplot2)}
\NormalTok{LL }\OtherTok{\textless{}{-}} \FunctionTok{ggplot}\NormalTok{(df\_bank, }\FunctionTok{aes}\NormalTok{(balance))}
\NormalTok{LL }\OtherTok{\textless{}{-}}\NormalTok{ LL }\SpecialCharTok{+} \FunctionTok{geom\_histogram}\NormalTok{(}\AttributeTok{stat=}\StringTok{"count"}\NormalTok{) }\SpecialCharTok{+} \FunctionTok{labs}\NormalTok{(}\AttributeTok{title =} \StringTok{"Balance"}\NormalTok{) }\SpecialCharTok{+}
  \FunctionTok{theme}\NormalTok{(}\AttributeTok{axis.text.x=}\FunctionTok{element\_text}\NormalTok{(}\AttributeTok{angle=}\DecValTok{90}\NormalTok{,}\AttributeTok{hjust=}\DecValTok{1}\NormalTok{,}\AttributeTok{vjust=}\FloatTok{0.5}\NormalTok{)) }\SpecialCharTok{+} 
  \FunctionTok{xlim}\NormalTok{(}\FunctionTok{c}\NormalTok{(}\DecValTok{0}\NormalTok{,}\DecValTok{20000}\NormalTok{)) }\SpecialCharTok{+} \FunctionTok{ylim}\NormalTok{(}\FunctionTok{c}\NormalTok{(}\DecValTok{0}\NormalTok{,}\DecValTok{25}\NormalTok{))}
\end{Highlighting}
\end{Shaded}

\begin{verbatim}
## Warning in geom_histogram(stat = "count"): Ignoring unknown parameters:
## `binwidth`, `bins`, and `pad`
\end{verbatim}

\begin{Shaded}
\begin{Highlighting}[]
\NormalTok{LL}
\end{Highlighting}
\end{Shaded}

\begin{verbatim}
## Warning: Removed 386 rows containing non-finite values (`stat_count()`).
\end{verbatim}

\begin{verbatim}
## Warning: Removed 1 rows containing missing values (`geom_bar()`).
\end{verbatim}

\includegraphics{Yenal_Arora_MATH-60604A_Final_Project_Solutions_files/figure-latex/unnamed-chunk-52-1.pdf}

\begin{Shaded}
\begin{Highlighting}[]
\FunctionTok{median}\NormalTok{(df\_bank}\SpecialCharTok{$}\NormalTok{balance)}
\end{Highlighting}
\end{Shaded}

\begin{verbatim}
## [1] 444
\end{verbatim}

The distribution here is very skewed to the right as we can see. We have
very high values as well which suggests we will have to deal with
outliers. Compared to all the values, the median here is only 444 which
is close to zero which suggests that most of the individuals contacted
have close to zero balance.

\begin{Shaded}
\begin{Highlighting}[]
\CommentTok{\#Campaign}

\NormalTok{MM }\OtherTok{\textless{}{-}} \FunctionTok{ggplot}\NormalTok{(df\_bank, }\FunctionTok{aes}\NormalTok{(campaign))}
\NormalTok{MM }\OtherTok{\textless{}{-}}\NormalTok{ MM }\SpecialCharTok{+} \FunctionTok{geom\_histogram}\NormalTok{(}\AttributeTok{stat=}\StringTok{"count"}\NormalTok{) }\SpecialCharTok{+} \FunctionTok{labs}\NormalTok{(}\AttributeTok{title =} \StringTok{"Campaign"}\NormalTok{)}\SpecialCharTok{+}
  \FunctionTok{theme}\NormalTok{(}\AttributeTok{axis.text.x=}\FunctionTok{element\_text}\NormalTok{(}\AttributeTok{angle=}\DecValTok{90}\NormalTok{,}\AttributeTok{hjust=}\DecValTok{1}\NormalTok{,}\AttributeTok{vjust=}\FloatTok{0.5}\NormalTok{))}
\end{Highlighting}
\end{Shaded}

\begin{verbatim}
## Warning in geom_histogram(stat = "count"): Ignoring unknown parameters:
## `binwidth`, `bins`, and `pad`
\end{verbatim}

\begin{Shaded}
\begin{Highlighting}[]
\NormalTok{MM}
\end{Highlighting}
\end{Shaded}

\includegraphics{Yenal_Arora_MATH-60604A_Final_Project_Solutions_files/figure-latex/unnamed-chunk-53-1.pdf}

The distribution here is again skewed to the right with majority of the
individuals being contacted only once or twice during this campaign.

\begin{Shaded}
\begin{Highlighting}[]
\CommentTok{\#Day}

\NormalTok{NN }\OtherTok{\textless{}{-}} \FunctionTok{ggplot}\NormalTok{(df\_bank, }\FunctionTok{aes}\NormalTok{(day))}
\NormalTok{NN }\OtherTok{\textless{}{-}}\NormalTok{ NN }\SpecialCharTok{+} \FunctionTok{geom\_histogram}\NormalTok{(}\AttributeTok{stat=}\StringTok{"count"}\NormalTok{) }\SpecialCharTok{+} \FunctionTok{labs}\NormalTok{(}\AttributeTok{title =} \StringTok{"Day"}\NormalTok{)}\SpecialCharTok{+}
  \FunctionTok{theme}\NormalTok{(}\AttributeTok{axis.text.x=}\FunctionTok{element\_text}\NormalTok{(}\AttributeTok{angle=}\DecValTok{90}\NormalTok{,}\AttributeTok{hjust=}\DecValTok{1}\NormalTok{,}\AttributeTok{vjust=}\FloatTok{0.5}\NormalTok{))}
\end{Highlighting}
\end{Shaded}

\begin{verbatim}
## Warning in geom_histogram(stat = "count"): Ignoring unknown parameters:
## `binwidth`, `bins`, and `pad`
\end{verbatim}

\begin{Shaded}
\begin{Highlighting}[]
\NormalTok{NN}
\end{Highlighting}
\end{Shaded}

\includegraphics{Yenal_Arora_MATH-60604A_Final_Project_Solutions_files/figure-latex/unnamed-chunk-54-1.pdf}

\begin{Shaded}
\begin{Highlighting}[]
\FunctionTok{mean}\NormalTok{(df\_bank}\SpecialCharTok{$}\NormalTok{day)}
\end{Highlighting}
\end{Shaded}

\begin{verbatim}
## [1] 15.91528
\end{verbatim}

\begin{Shaded}
\begin{Highlighting}[]
\FunctionTok{median}\NormalTok{(df\_bank}\SpecialCharTok{$}\NormalTok{day)}
\end{Highlighting}
\end{Shaded}

\begin{verbatim}
## [1] 16
\end{verbatim}

The distribution here is very uniform with mean of approx. 15.9 almost
equal to median of 16.

\begin{Shaded}
\begin{Highlighting}[]
\CommentTok{\#Duration}

\NormalTok{OO }\OtherTok{\textless{}{-}} \FunctionTok{ggplot}\NormalTok{(df\_bank, }\FunctionTok{aes}\NormalTok{(duration))}
\NormalTok{OO }\OtherTok{\textless{}{-}}\NormalTok{ OO }\SpecialCharTok{+} \FunctionTok{geom\_histogram}\NormalTok{(}\AttributeTok{stat=}\StringTok{"count"}\NormalTok{) }\SpecialCharTok{+} \FunctionTok{labs}\NormalTok{(}\AttributeTok{title =} \StringTok{"Duration"}\NormalTok{)}\SpecialCharTok{+}
  \FunctionTok{theme}\NormalTok{(}\AttributeTok{axis.text.x=}\FunctionTok{element\_text}\NormalTok{(}\AttributeTok{angle=}\DecValTok{90}\NormalTok{,}\AttributeTok{hjust=}\DecValTok{1}\NormalTok{,}\AttributeTok{vjust=}\FloatTok{0.5}\NormalTok{))}
\end{Highlighting}
\end{Shaded}

\begin{verbatim}
## Warning in geom_histogram(stat = "count"): Ignoring unknown parameters:
## `binwidth`, `bins`, and `pad`
\end{verbatim}

\begin{Shaded}
\begin{Highlighting}[]
\NormalTok{OO}
\end{Highlighting}
\end{Shaded}

\includegraphics{Yenal_Arora_MATH-60604A_Final_Project_Solutions_files/figure-latex/unnamed-chunk-55-1.pdf}

\begin{Shaded}
\begin{Highlighting}[]
\FunctionTok{median}\NormalTok{(df\_bank}\SpecialCharTok{$}\NormalTok{duration)}
\end{Highlighting}
\end{Shaded}

\begin{verbatim}
## [1] 185
\end{verbatim}

We have some big values here with a right skewed distribution. The
median here is 185 seconds (close to 3 minutes), which suggests that
most individuals decide about subscrbing to a term deposit or not within
the first 3 to 4 minutes of the call duration.

\begin{Shaded}
\begin{Highlighting}[]
\CommentTok{\#Pdays}

\NormalTok{PP }\OtherTok{\textless{}{-}} \FunctionTok{ggplot}\NormalTok{(df\_bank, }\FunctionTok{aes}\NormalTok{(pdays))}
\NormalTok{PP }\OtherTok{\textless{}{-}}\NormalTok{ PP }\SpecialCharTok{+} \FunctionTok{geom\_histogram}\NormalTok{(}\AttributeTok{stat=}\StringTok{"count"}\NormalTok{) }\SpecialCharTok{+} \FunctionTok{labs}\NormalTok{(}\AttributeTok{title =} \StringTok{"Pdays"}\NormalTok{)}\SpecialCharTok{+}
  \FunctionTok{theme}\NormalTok{(}\AttributeTok{axis.text.x=}\FunctionTok{element\_text}\NormalTok{(}\AttributeTok{angle=}\DecValTok{90}\NormalTok{,}\AttributeTok{hjust=}\DecValTok{1}\NormalTok{,}\AttributeTok{vjust=}\FloatTok{0.5}\NormalTok{))}
\end{Highlighting}
\end{Shaded}

\begin{verbatim}
## Warning in geom_histogram(stat = "count"): Ignoring unknown parameters:
## `binwidth`, `bins`, and `pad`
\end{verbatim}

\begin{Shaded}
\begin{Highlighting}[]
\NormalTok{PP}
\end{Highlighting}
\end{Shaded}

\includegraphics{Yenal_Arora_MATH-60604A_Final_Project_Solutions_files/figure-latex/unnamed-chunk-56-1.pdf}

\begin{Shaded}
\begin{Highlighting}[]
\FunctionTok{median}\NormalTok{(df\_bank}\SpecialCharTok{$}\NormalTok{pdays)}
\end{Highlighting}
\end{Shaded}

\begin{verbatim}
## [1] -1
\end{verbatim}

The distribution is extremely skewed with a median of perhaps -1. This
value means those individuals who have never been contacted before this
campaign, that is, most of the individuals were contacted for the very
first time during this campaign.

\begin{Shaded}
\begin{Highlighting}[]
\CommentTok{\#Previous}

\NormalTok{QQ }\OtherTok{\textless{}{-}} \FunctionTok{ggplot}\NormalTok{(df\_bank, }\FunctionTok{aes}\NormalTok{(previous))}
\NormalTok{QQ }\OtherTok{\textless{}{-}}\NormalTok{ QQ }\SpecialCharTok{+} \FunctionTok{geom\_histogram}\NormalTok{(}\AttributeTok{stat=}\StringTok{"count"}\NormalTok{) }\SpecialCharTok{+} \FunctionTok{labs}\NormalTok{(}\AttributeTok{title =} \StringTok{"Previous"}\NormalTok{)}\SpecialCharTok{+}
  \FunctionTok{theme}\NormalTok{(}\AttributeTok{axis.text.x=}\FunctionTok{element\_text}\NormalTok{(}\AttributeTok{angle=}\DecValTok{90}\NormalTok{,}\AttributeTok{hjust=}\DecValTok{1}\NormalTok{,}\AttributeTok{vjust=}\FloatTok{0.5}\NormalTok{))}
\end{Highlighting}
\end{Shaded}

\begin{verbatim}
## Warning in geom_histogram(stat = "count"): Ignoring unknown parameters:
## `binwidth`, `bins`, and `pad`
\end{verbatim}

\begin{Shaded}
\begin{Highlighting}[]
\NormalTok{QQ}
\end{Highlighting}
\end{Shaded}

\includegraphics{Yenal_Arora_MATH-60604A_Final_Project_Solutions_files/figure-latex/unnamed-chunk-57-1.pdf}

\begin{Shaded}
\begin{Highlighting}[]
\FunctionTok{median}\NormalTok{(df\_bank}\SpecialCharTok{$}\NormalTok{previous)}
\end{Highlighting}
\end{Shaded}

\begin{verbatim}
## [1] 0
\end{verbatim}

This corroborates with the previous graph (pdays) which suggests that
there was no communication before with the individuals contacted during
this campaign (Median = 0).

\textbf{Analysis of the independent variables in relation to the
response variable}

\begin{Shaded}
\begin{Highlighting}[]
\CommentTok{\# Job and y}

\FunctionTok{library}\NormalTok{(ggplot2)}
\NormalTok{A }\OtherTok{\textless{}{-}} \FunctionTok{ggplot}\NormalTok{(df\_bank, }\FunctionTok{aes}\NormalTok{(job,}\AttributeTok{fill =}\NormalTok{ y))}
\NormalTok{A }\OtherTok{\textless{}{-}}\NormalTok{ A }\SpecialCharTok{+} \FunctionTok{geom\_histogram}\NormalTok{(}\AttributeTok{stat=}\StringTok{"count"}\NormalTok{) }\SpecialCharTok{+} \FunctionTok{labs}\NormalTok{(}\AttributeTok{title =} \StringTok{"job and y"}\NormalTok{) }\SpecialCharTok{+}
  \FunctionTok{theme}\NormalTok{(}\AttributeTok{axis.text.x=}\FunctionTok{element\_text}\NormalTok{(}\AttributeTok{angle=}\DecValTok{90}\NormalTok{,}\AttributeTok{hjust=}\DecValTok{1}\NormalTok{,}\AttributeTok{vjust=}\FloatTok{0.5}\NormalTok{))}
\end{Highlighting}
\end{Shaded}

\begin{verbatim}
## Warning in geom_histogram(stat = "count"): Ignoring unknown parameters:
## `binwidth`, `bins`, and `pad`
\end{verbatim}

\begin{Shaded}
\begin{Highlighting}[]
\NormalTok{A}
\end{Highlighting}
\end{Shaded}

\includegraphics{Yenal_Arora_MATH-60604A_Final_Project_Solutions_files/figure-latex/unnamed-chunk-58-1.pdf}

We can see that individuals with management job are the ones who have
subscribed the maximum to a term deposit followed by technicians.

\begin{Shaded}
\begin{Highlighting}[]
\CommentTok{\#Marital and y}

\NormalTok{B }\OtherTok{\textless{}{-}} \FunctionTok{ggplot}\NormalTok{(df\_bank, }\FunctionTok{aes}\NormalTok{(marital,}\AttributeTok{fill =}\NormalTok{ y))}
\NormalTok{B }\OtherTok{\textless{}{-}}\NormalTok{ B }\SpecialCharTok{+} \FunctionTok{geom\_histogram}\NormalTok{(}\AttributeTok{stat=}\StringTok{"count"}\NormalTok{) }\SpecialCharTok{+} \FunctionTok{labs}\NormalTok{(}\AttributeTok{title =} \StringTok{"marital and y"}\NormalTok{) }\SpecialCharTok{+}
  \FunctionTok{theme}\NormalTok{(}\AttributeTok{axis.text.x=}\FunctionTok{element\_text}\NormalTok{(}\AttributeTok{angle=}\DecValTok{90}\NormalTok{,}\AttributeTok{hjust=}\DecValTok{1}\NormalTok{,}\AttributeTok{vjust=}\FloatTok{0.5}\NormalTok{))}
\end{Highlighting}
\end{Shaded}

\begin{verbatim}
## Warning in geom_histogram(stat = "count"): Ignoring unknown parameters:
## `binwidth`, `bins`, and `pad`
\end{verbatim}

\begin{Shaded}
\begin{Highlighting}[]
\NormalTok{B}
\end{Highlighting}
\end{Shaded}

\includegraphics{Yenal_Arora_MATH-60604A_Final_Project_Solutions_files/figure-latex/unnamed-chunk-59-1.pdf}

Married individuals have more tendency to subscribe to a term deposit.

\begin{Shaded}
\begin{Highlighting}[]
\CommentTok{\#Education and y}

\NormalTok{C }\OtherTok{\textless{}{-}} \FunctionTok{ggplot}\NormalTok{(df\_bank, }\FunctionTok{aes}\NormalTok{(education,}\AttributeTok{fill =}\NormalTok{ y))}
\NormalTok{C }\OtherTok{\textless{}{-}}\NormalTok{ C }\SpecialCharTok{+} \FunctionTok{geom\_histogram}\NormalTok{(}\AttributeTok{stat=}\StringTok{"count"}\NormalTok{) }\SpecialCharTok{+} \FunctionTok{labs}\NormalTok{(}\AttributeTok{title =} \StringTok{"education and y"}\NormalTok{) }\SpecialCharTok{+}
  \FunctionTok{theme}\NormalTok{(}\AttributeTok{axis.text.x=}\FunctionTok{element\_text}\NormalTok{(}\AttributeTok{angle=}\DecValTok{90}\NormalTok{,}\AttributeTok{hjust=}\DecValTok{1}\NormalTok{,}\AttributeTok{vjust=}\FloatTok{0.5}\NormalTok{))}
\end{Highlighting}
\end{Shaded}

\begin{verbatim}
## Warning in geom_histogram(stat = "count"): Ignoring unknown parameters:
## `binwidth`, `bins`, and `pad`
\end{verbatim}

\begin{Shaded}
\begin{Highlighting}[]
\NormalTok{C}
\end{Highlighting}
\end{Shaded}

\includegraphics{Yenal_Arora_MATH-60604A_Final_Project_Solutions_files/figure-latex/unnamed-chunk-60-1.pdf}

Individuals with secondary education subscribe the most to a term
deposit.

\begin{Shaded}
\begin{Highlighting}[]
\CommentTok{\#Default and y}

\NormalTok{D }\OtherTok{\textless{}{-}} \FunctionTok{ggplot}\NormalTok{(df\_bank, }\FunctionTok{aes}\NormalTok{(default,}\AttributeTok{fill =}\NormalTok{ y))}
\NormalTok{D }\OtherTok{\textless{}{-}}\NormalTok{ D }\SpecialCharTok{+} \FunctionTok{geom\_histogram}\NormalTok{(}\AttributeTok{stat=}\StringTok{"count"}\NormalTok{) }\SpecialCharTok{+} \FunctionTok{labs}\NormalTok{(}\AttributeTok{title =} \StringTok{"default and y"}\NormalTok{) }\SpecialCharTok{+}
  \FunctionTok{theme}\NormalTok{(}\AttributeTok{axis.text.x=}\FunctionTok{element\_text}\NormalTok{(}\AttributeTok{angle=}\DecValTok{90}\NormalTok{,}\AttributeTok{hjust=}\DecValTok{1}\NormalTok{,}\AttributeTok{vjust=}\FloatTok{0.5}\NormalTok{))}
\end{Highlighting}
\end{Shaded}

\begin{verbatim}
## Warning in geom_histogram(stat = "count"): Ignoring unknown parameters:
## `binwidth`, `bins`, and `pad`
\end{verbatim}

\begin{Shaded}
\begin{Highlighting}[]
\NormalTok{D}
\end{Highlighting}
\end{Shaded}

\includegraphics{Yenal_Arora_MATH-60604A_Final_Project_Solutions_files/figure-latex/unnamed-chunk-61-1.pdf}

Individuals who have no credits in default subscribe to a term deposit
whereas for individulas who have credits in default hardly subscribe.

\begin{Shaded}
\begin{Highlighting}[]
\CommentTok{\#Housing and y}

\NormalTok{E }\OtherTok{\textless{}{-}} \FunctionTok{ggplot}\NormalTok{(df\_bank, }\FunctionTok{aes}\NormalTok{(housing,}\AttributeTok{fill =}\NormalTok{ y))}
\NormalTok{E }\OtherTok{\textless{}{-}}\NormalTok{ E }\SpecialCharTok{+} \FunctionTok{geom\_histogram}\NormalTok{(}\AttributeTok{stat=}\StringTok{"count"}\NormalTok{) }\SpecialCharTok{+} \FunctionTok{labs}\NormalTok{(}\AttributeTok{title =} \StringTok{"housing and y"}\NormalTok{) }\SpecialCharTok{+}
  \FunctionTok{theme}\NormalTok{(}\AttributeTok{axis.text.x=}\FunctionTok{element\_text}\NormalTok{(}\AttributeTok{angle=}\DecValTok{90}\NormalTok{,}\AttributeTok{hjust=}\DecValTok{1}\NormalTok{,}\AttributeTok{vjust=}\FloatTok{0.5}\NormalTok{))}
\end{Highlighting}
\end{Shaded}

\begin{verbatim}
## Warning in geom_histogram(stat = "count"): Ignoring unknown parameters:
## `binwidth`, `bins`, and `pad`
\end{verbatim}

\begin{Shaded}
\begin{Highlighting}[]
\NormalTok{E}
\end{Highlighting}
\end{Shaded}

\includegraphics{Yenal_Arora_MATH-60604A_Final_Project_Solutions_files/figure-latex/unnamed-chunk-62-1.pdf}

Individuals who do not have a housing loan tend to subsribe more to a
term deposit compared to the individuals who have a housing loan.

\begin{Shaded}
\begin{Highlighting}[]
\CommentTok{\#Loan and y}

\NormalTok{F }\OtherTok{\textless{}{-}} \FunctionTok{ggplot}\NormalTok{(df\_bank, }\FunctionTok{aes}\NormalTok{(loan,}\AttributeTok{fill =}\NormalTok{ y))}
\NormalTok{F }\OtherTok{\textless{}{-}}\NormalTok{ F }\SpecialCharTok{+} \FunctionTok{geom\_histogram}\NormalTok{(}\AttributeTok{stat=}\StringTok{"count"}\NormalTok{) }\SpecialCharTok{+} \FunctionTok{labs}\NormalTok{(}\AttributeTok{title =} \StringTok{"loan and y"}\NormalTok{) }\SpecialCharTok{+}
  \FunctionTok{theme}\NormalTok{(}\AttributeTok{axis.text.x=}\FunctionTok{element\_text}\NormalTok{(}\AttributeTok{angle=}\DecValTok{90}\NormalTok{,}\AttributeTok{hjust=}\DecValTok{1}\NormalTok{,}\AttributeTok{vjust=}\FloatTok{0.5}\NormalTok{))}
\end{Highlighting}
\end{Shaded}

\begin{verbatim}
## Warning in geom_histogram(stat = "count"): Ignoring unknown parameters:
## `binwidth`, `bins`, and `pad`
\end{verbatim}

\begin{Shaded}
\begin{Highlighting}[]
\NormalTok{F}
\end{Highlighting}
\end{Shaded}

\includegraphics{Yenal_Arora_MATH-60604A_Final_Project_Solutions_files/figure-latex/unnamed-chunk-63-1.pdf}

Individuals who do not have a personal loan subscribe more to a term
deposit.

\begin{Shaded}
\begin{Highlighting}[]
\CommentTok{\#Contact and y}

\NormalTok{G }\OtherTok{\textless{}{-}} \FunctionTok{ggplot}\NormalTok{(df\_bank, }\FunctionTok{aes}\NormalTok{(contact,}\AttributeTok{fill =}\NormalTok{ y))}
\NormalTok{G }\OtherTok{\textless{}{-}}\NormalTok{ G }\SpecialCharTok{+} \FunctionTok{geom\_histogram}\NormalTok{(}\AttributeTok{stat=}\StringTok{"count"}\NormalTok{) }\SpecialCharTok{+} \FunctionTok{labs}\NormalTok{(}\AttributeTok{title =} \StringTok{"contact and y"}\NormalTok{) }\SpecialCharTok{+}
  \FunctionTok{theme}\NormalTok{(}\AttributeTok{axis.text.x=}\FunctionTok{element\_text}\NormalTok{(}\AttributeTok{angle=}\DecValTok{90}\NormalTok{,}\AttributeTok{hjust=}\DecValTok{1}\NormalTok{,}\AttributeTok{vjust=}\FloatTok{0.5}\NormalTok{))}
\end{Highlighting}
\end{Shaded}

\begin{verbatim}
## Warning in geom_histogram(stat = "count"): Ignoring unknown parameters:
## `binwidth`, `bins`, and `pad`
\end{verbatim}

\begin{Shaded}
\begin{Highlighting}[]
\NormalTok{G}
\end{Highlighting}
\end{Shaded}

\includegraphics{Yenal_Arora_MATH-60604A_Final_Project_Solutions_files/figure-latex/unnamed-chunk-64-1.pdf}

When the contact communication type was Cellular, individuals subscribed
more to the bank's term deposit.

\begin{Shaded}
\begin{Highlighting}[]
\CommentTok{\#poutcome and y}

\NormalTok{H }\OtherTok{\textless{}{-}} \FunctionTok{ggplot}\NormalTok{(df\_bank, }\FunctionTok{aes}\NormalTok{(poutcome,}\AttributeTok{fill =}\NormalTok{ y))}
\NormalTok{H }\OtherTok{\textless{}{-}}\NormalTok{ H }\SpecialCharTok{+} \FunctionTok{geom\_histogram}\NormalTok{(}\AttributeTok{stat=}\StringTok{"count"}\NormalTok{) }\SpecialCharTok{+} \FunctionTok{labs}\NormalTok{(}\AttributeTok{title =} \StringTok{"poutcome and y"}\NormalTok{) }\SpecialCharTok{+}
  \FunctionTok{theme}\NormalTok{(}\AttributeTok{axis.text.x=}\FunctionTok{element\_text}\NormalTok{(}\AttributeTok{angle=}\DecValTok{90}\NormalTok{,}\AttributeTok{hjust=}\DecValTok{1}\NormalTok{,}\AttributeTok{vjust=}\FloatTok{0.5}\NormalTok{))}
\end{Highlighting}
\end{Shaded}

\begin{verbatim}
## Warning in geom_histogram(stat = "count"): Ignoring unknown parameters:
## `binwidth`, `bins`, and `pad`
\end{verbatim}

\begin{Shaded}
\begin{Highlighting}[]
\NormalTok{H}
\end{Highlighting}
\end{Shaded}

\includegraphics{Yenal_Arora_MATH-60604A_Final_Project_Solutions_files/figure-latex/unnamed-chunk-65-1.pdf}

When the outcome of the previous marketing campaign was a success,
individuals subscribe more to the term deposit. However, as mentioned
before, the unknown category might represent the new individuals and
they are the ones who subscribe the most to the term deposits.
Therefore, it's best to keep this category in for the time being we
reach the modelling stage.

\begin{Shaded}
\begin{Highlighting}[]
\CommentTok{\#age and y}

\NormalTok{I }\OtherTok{\textless{}{-}} \FunctionTok{ggplot}\NormalTok{(df\_bank, }\FunctionTok{aes}\NormalTok{(age,}\AttributeTok{fill =}\NormalTok{ y))}
\NormalTok{I }\OtherTok{\textless{}{-}}\NormalTok{ I }\SpecialCharTok{+} \FunctionTok{geom\_histogram}\NormalTok{(}\AttributeTok{stat=}\StringTok{"count"}\NormalTok{) }\SpecialCharTok{+} \FunctionTok{labs}\NormalTok{(}\AttributeTok{title =} \StringTok{"age and y"}\NormalTok{) }\SpecialCharTok{+}
  \FunctionTok{theme}\NormalTok{(}\AttributeTok{axis.text.x=}\FunctionTok{element\_text}\NormalTok{(}\AttributeTok{angle=}\DecValTok{90}\NormalTok{,}\AttributeTok{hjust=}\DecValTok{1}\NormalTok{,}\AttributeTok{vjust=}\FloatTok{0.5}\NormalTok{))}
\end{Highlighting}
\end{Shaded}

\begin{verbatim}
## Warning in geom_histogram(stat = "count"): Ignoring unknown parameters:
## `binwidth`, `bins`, and `pad`
\end{verbatim}

\begin{Shaded}
\begin{Highlighting}[]
\NormalTok{I}
\end{Highlighting}
\end{Shaded}

\includegraphics{Yenal_Arora_MATH-60604A_Final_Project_Solutions_files/figure-latex/unnamed-chunk-66-1.pdf}

Individuals of age 25-45 years old subscribe the most to term deposits.
The same age group has also the highest counts of not subscribing to the
term deposit. This suggests that individuals in this age range are
contacted the most due to their high presence in our dataset.

\begin{Shaded}
\begin{Highlighting}[]
\CommentTok{\#balance and y}
\FunctionTok{library}\NormalTok{(dplyr)}
\NormalTok{y\_yes }\OtherTok{\textless{}{-}}\NormalTok{ df\_bank }\SpecialCharTok{\%\textgreater{}\%} \FunctionTok{filter}\NormalTok{(df\_bank}\SpecialCharTok{$}\NormalTok{y }\SpecialCharTok{==}\StringTok{"yes"}\NormalTok{)}
\NormalTok{y\_no }\OtherTok{\textless{}{-}}\NormalTok{ df\_bank }\SpecialCharTok{\%\textgreater{}\%} \FunctionTok{filter}\NormalTok{(df\_bank}\SpecialCharTok{$}\NormalTok{y }\SpecialCharTok{==}\StringTok{"no"}\NormalTok{)}

\NormalTok{J }\OtherTok{\textless{}{-}} \FunctionTok{ggplot}\NormalTok{(y\_yes, }\FunctionTok{aes}\NormalTok{(balance))}
\NormalTok{J }\OtherTok{\textless{}{-}}\NormalTok{ J }\SpecialCharTok{+} \FunctionTok{geom\_histogram}\NormalTok{(}\AttributeTok{stat=}\StringTok{"count"}\NormalTok{, }\AttributeTok{binwidth=}\DecValTok{10}\NormalTok{) }\SpecialCharTok{+} \FunctionTok{labs}\NormalTok{(}\AttributeTok{title =} \StringTok{"balance and y\_yes"}\NormalTok{) }\SpecialCharTok{+}
  \FunctionTok{theme}\NormalTok{(}\AttributeTok{axis.text.x=}\FunctionTok{element\_text}\NormalTok{(}\AttributeTok{angle=}\DecValTok{90}\NormalTok{,}\AttributeTok{hjust=}\DecValTok{1}\NormalTok{,}\AttributeTok{vjust=}\FloatTok{0.5}\NormalTok{)) }\SpecialCharTok{+} 
  \FunctionTok{xlim}\NormalTok{(}\FunctionTok{c}\NormalTok{(}\DecValTok{0}\NormalTok{,}\DecValTok{3000}\NormalTok{)) }\SpecialCharTok{+} \FunctionTok{ylim}\NormalTok{(}\FunctionTok{c}\NormalTok{(}\DecValTok{0}\NormalTok{,}\DecValTok{25}\NormalTok{))}
\end{Highlighting}
\end{Shaded}

\begin{verbatim}
## Warning in geom_histogram(stat = "count", binwidth = 10): Ignoring unknown
## parameters: `binwidth`, `bins`, and `pad`
\end{verbatim}

\begin{Shaded}
\begin{Highlighting}[]
\NormalTok{J}
\end{Highlighting}
\end{Shaded}

\begin{verbatim}
## Warning: Removed 120 rows containing non-finite values (`stat_count()`).
\end{verbatim}

\begin{verbatim}
## Warning: Removed 1 rows containing missing values (`geom_bar()`).
\end{verbatim}

\includegraphics{Yenal_Arora_MATH-60604A_Final_Project_Solutions_files/figure-latex/unnamed-chunk-67-1.pdf}

\begin{Shaded}
\begin{Highlighting}[]
\NormalTok{K }\OtherTok{\textless{}{-}} \FunctionTok{ggplot}\NormalTok{(y\_no, }\FunctionTok{aes}\NormalTok{(balance))}
\NormalTok{K }\OtherTok{\textless{}{-}}\NormalTok{ K }\SpecialCharTok{+} \FunctionTok{geom\_histogram}\NormalTok{(}\AttributeTok{stat=}\StringTok{"count"}\NormalTok{, }\AttributeTok{binwidth=}\DecValTok{10}\NormalTok{) }\SpecialCharTok{+} \FunctionTok{labs}\NormalTok{(}\AttributeTok{title =} \StringTok{"balance and y\_no"}\NormalTok{) }\SpecialCharTok{+}
  \FunctionTok{theme}\NormalTok{(}\AttributeTok{axis.text.x=}\FunctionTok{element\_text}\NormalTok{(}\AttributeTok{angle=}\DecValTok{90}\NormalTok{,}\AttributeTok{hjust=}\DecValTok{1}\NormalTok{,}\AttributeTok{vjust=}\FloatTok{0.5}\NormalTok{)) }\SpecialCharTok{+} 
  \FunctionTok{xlim}\NormalTok{(}\FunctionTok{c}\NormalTok{(}\DecValTok{0}\NormalTok{,}\DecValTok{3000}\NormalTok{)) }\SpecialCharTok{+} \FunctionTok{ylim}\NormalTok{(}\FunctionTok{c}\NormalTok{(}\DecValTok{0}\NormalTok{,}\DecValTok{25}\NormalTok{))}
\end{Highlighting}
\end{Shaded}

\begin{verbatim}
## Warning in geom_histogram(stat = "count", binwidth = 10): Ignoring unknown
## parameters: `binwidth`, `bins`, and `pad`
\end{verbatim}

\begin{Shaded}
\begin{Highlighting}[]
\NormalTok{K}
\end{Highlighting}
\end{Shaded}

\begin{verbatim}
## Warning: Removed 865 rows containing non-finite values (`stat_count()`).
## Removed 1 rows containing missing values (`geom_bar()`).
\end{verbatim}

\includegraphics{Yenal_Arora_MATH-60604A_Final_Project_Solutions_files/figure-latex/unnamed-chunk-67-2.pdf}

Overall, these two graphs show that individuals who have very low
balance don't usually subscribe to a term deposit and very very few
people with a low balance actually subscribe for the deposit.

\begin{Shaded}
\begin{Highlighting}[]
\CommentTok{\#duration and y}

\NormalTok{L }\OtherTok{\textless{}{-}} \FunctionTok{ggplot}\NormalTok{(df\_bank, }\FunctionTok{aes}\NormalTok{(duration,}\AttributeTok{fill =}\NormalTok{ y))}
\NormalTok{L }\OtherTok{\textless{}{-}}\NormalTok{ L }\SpecialCharTok{+} \FunctionTok{geom\_histogram}\NormalTok{(}\AttributeTok{stat=}\StringTok{"count"}\NormalTok{) }\SpecialCharTok{+} \FunctionTok{labs}\NormalTok{(}\AttributeTok{title =} \StringTok{"duration and y"}\NormalTok{) }\SpecialCharTok{+}
  \FunctionTok{theme}\NormalTok{(}\AttributeTok{axis.text.x=}\FunctionTok{element\_text}\NormalTok{(}\AttributeTok{angle=}\DecValTok{90}\NormalTok{,}\AttributeTok{hjust=}\DecValTok{1}\NormalTok{,}\AttributeTok{vjust=}\FloatTok{0.5}\NormalTok{))}\SpecialCharTok{+} 
  \FunctionTok{xlim}\NormalTok{(}\FunctionTok{c}\NormalTok{(}\DecValTok{0}\NormalTok{,}\DecValTok{1000}\NormalTok{))}
\end{Highlighting}
\end{Shaded}

\begin{verbatim}
## Warning in geom_histogram(stat = "count"): Ignoring unknown parameters:
## `binwidth`, `bins`, and `pad`
\end{verbatim}

\begin{Shaded}
\begin{Highlighting}[]
\NormalTok{L}
\end{Highlighting}
\end{Shaded}

\begin{verbatim}
## Warning: Removed 107 rows containing non-finite values (`stat_count()`).
\end{verbatim}

\includegraphics{Yenal_Arora_MATH-60604A_Final_Project_Solutions_files/figure-latex/unnamed-chunk-68-1.pdf}

We can observe from the graph that the individuals who do not wish to
subscribe for the term deposit are able to decide the same within the
first few minutes of the call and those who wish to subscribe for the
deposit usually take longer.

This variable has an interesting impact as it highly affects the the
response variable. When the duration = 0, then y is always = no
i.e.~that is there is no term deposit subscription. Yet, the duration is
not known before making the call. Also, after the end of the call, y is
obviously known. Therefore, in order to have a realistic predictive
model, this variable should be discarded from the analysis.

\begin{Shaded}
\begin{Highlighting}[]
\CommentTok{\#month and y}

\NormalTok{M }\OtherTok{\textless{}{-}} \FunctionTok{ggplot}\NormalTok{(df\_bank, }\FunctionTok{aes}\NormalTok{(month,}\AttributeTok{fill =}\NormalTok{ y))}
\NormalTok{M }\OtherTok{\textless{}{-}}\NormalTok{ M }\SpecialCharTok{+} \FunctionTok{geom\_histogram}\NormalTok{(}\AttributeTok{stat=}\StringTok{"count"}\NormalTok{) }\SpecialCharTok{+} \FunctionTok{labs}\NormalTok{(}\AttributeTok{title =} \StringTok{"month and y"}\NormalTok{) }\SpecialCharTok{+}
  \FunctionTok{theme}\NormalTok{(}\AttributeTok{axis.text.x=}\FunctionTok{element\_text}\NormalTok{(}\AttributeTok{angle=}\DecValTok{90}\NormalTok{,}\AttributeTok{hjust=}\DecValTok{1}\NormalTok{,}\AttributeTok{vjust=}\FloatTok{0.5}\NormalTok{))}
\end{Highlighting}
\end{Shaded}

\begin{verbatim}
## Warning in geom_histogram(stat = "count"): Ignoring unknown parameters:
## `binwidth`, `bins`, and `pad`
\end{verbatim}

\begin{Shaded}
\begin{Highlighting}[]
\NormalTok{M}
\end{Highlighting}
\end{Shaded}

\includegraphics{Yenal_Arora_MATH-60604A_Final_Project_Solutions_files/figure-latex/unnamed-chunk-69-1.pdf}

In the months of May and August, individuals tend to subscribe more to
term deposits.

We can similarly check the relation of other numerical variables such as
previous, campaign, pdays with the response variable y.

\textbf{Treatment of outliers}

From the analysis done before, we saw that some of the variables have
outliers and it is important to deal with them before we proceed with
the actual modelling. A very common method of outliers which we can use
here is capping where the outliers below the 5th percentile and above
the 95th percentile are capped.

As observed, we can perform this capping on the following numeric
variables: balance, campaign, pdays and previous.

\begin{Shaded}
\begin{Highlighting}[]
\FunctionTok{library}\NormalTok{(dlookr)}
\NormalTok{df\_bank}\SpecialCharTok{$}\NormalTok{balance }\OtherTok{\textless{}{-}} \FunctionTok{imputate\_outlier}\NormalTok{(df\_bank, balance, }\AttributeTok{method =} \StringTok{"capping"}\NormalTok{)}
\NormalTok{df\_bank}\SpecialCharTok{$}\NormalTok{campaign }\OtherTok{\textless{}{-}} \FunctionTok{imputate\_outlier}\NormalTok{(df\_bank, campaign, }\AttributeTok{method =} \StringTok{"capping"}\NormalTok{)}
\NormalTok{df\_bank}\SpecialCharTok{$}\NormalTok{pdays }\OtherTok{\textless{}{-}} \FunctionTok{imputate\_outlier}\NormalTok{(df\_bank, pdays, }\AttributeTok{method =} \StringTok{"capping"}\NormalTok{)}
\NormalTok{df\_bank}\SpecialCharTok{$}\NormalTok{previous }\OtherTok{\textless{}{-}} \FunctionTok{imputate\_outlier}\NormalTok{(df\_bank, previous, }\AttributeTok{method =} \StringTok{"capping"}\NormalTok{)}
\end{Highlighting}
\end{Shaded}

\textbf{Treatment of Categorical Variables}

As seen before, for some of the categorical variables (Marital, Month
and Job), the distribution could be improved to have a better predictive
performance if we group some of the similar levels together into new
broader levels.

\begin{Shaded}
\begin{Highlighting}[]
\FunctionTok{library}\NormalTok{(rockchalk)}
\end{Highlighting}
\end{Shaded}

\begin{verbatim}
## 
## Attaching package: 'rockchalk'
\end{verbatim}

\begin{verbatim}
## The following objects are masked from 'package:dlookr':
## 
##     kurtosis, skewness
\end{verbatim}

\begin{verbatim}
## The following object is masked from 'package:dplyr':
## 
##     summarize
\end{verbatim}

\begin{Shaded}
\begin{Highlighting}[]
\NormalTok{df\_bank}\SpecialCharTok{$}\NormalTok{marital }\OtherTok{\textless{}{-}} \FunctionTok{combineLevels}\NormalTok{(df\_bank}\SpecialCharTok{$}\NormalTok{marital, }\AttributeTok{levs =} \FunctionTok{c}\NormalTok{(}\StringTok{"single"}\NormalTok{, }\StringTok{"divorced"}\NormalTok{),}
                                 \AttributeTok{newLabel =} \StringTok{"unmarried"}\NormalTok{)}

\NormalTok{df\_bank}\SpecialCharTok{$}\NormalTok{month }\OtherTok{\textless{}{-}} \FunctionTok{combineLevels}\NormalTok{(df\_bank}\SpecialCharTok{$}\NormalTok{month, }\AttributeTok{levs =} \FunctionTok{c}\NormalTok{(}\StringTok{"aug"}\NormalTok{, }\StringTok{"sep"}\NormalTok{,}\StringTok{"oct"}\NormalTok{), }
                               \AttributeTok{newLabel =} \StringTok{"Autumn"}\NormalTok{)}
\NormalTok{df\_bank}\SpecialCharTok{$}\NormalTok{month }\OtherTok{\textless{}{-}} \FunctionTok{combineLevels}\NormalTok{(df\_bank}\SpecialCharTok{$}\NormalTok{month, }\AttributeTok{levs =} \FunctionTok{c}\NormalTok{(}\StringTok{"nov"}\NormalTok{, }\StringTok{"dec"}\NormalTok{, }\StringTok{"jan"}\NormalTok{, }
                                                       \StringTok{"feb"}\NormalTok{, }\StringTok{"mar"}\NormalTok{), }
                               \AttributeTok{newLabel =} \StringTok{"Winter"}\NormalTok{)}
\NormalTok{df\_bank}\SpecialCharTok{$}\NormalTok{month }\OtherTok{\textless{}{-}} \FunctionTok{combineLevels}\NormalTok{(df\_bank}\SpecialCharTok{$}\NormalTok{month, }\AttributeTok{levs =} \FunctionTok{c}\NormalTok{(}\StringTok{"apr"}\NormalTok{,}\StringTok{"may"}\NormalTok{, }\StringTok{"jun"}\NormalTok{, }\StringTok{"jul"}\NormalTok{),}
                               \AttributeTok{newLabel =} \StringTok{"Summer"}\NormalTok{)}

\NormalTok{df\_bank}\SpecialCharTok{$}\NormalTok{job }\OtherTok{\textless{}{-}} \FunctionTok{combineLevels}\NormalTok{(df\_bank}\SpecialCharTok{$}\NormalTok{job, }\AttributeTok{levs =} \FunctionTok{c}\NormalTok{(}\StringTok{"unemployed"}\NormalTok{, }
                                                   \StringTok{"retired"}\NormalTok{, }\StringTok{"student"}\NormalTok{), }
                             \AttributeTok{newLabel =} \StringTok{"Not Employed"}\NormalTok{)}
\NormalTok{df\_bank}\SpecialCharTok{$}\NormalTok{job }\OtherTok{\textless{}{-}} \FunctionTok{combineLevels}\NormalTok{(df\_bank}\SpecialCharTok{$}\NormalTok{job, }\AttributeTok{levs =} \FunctionTok{c}\NormalTok{(}\StringTok{"admin."}\NormalTok{, }\StringTok{"management"}\NormalTok{), }
                             \AttributeTok{newLabel =} \StringTok{"admin and mgnt"}\NormalTok{)}
\NormalTok{df\_bank}\SpecialCharTok{$}\NormalTok{job }\OtherTok{\textless{}{-}} \FunctionTok{combineLevels}\NormalTok{(df\_bank}\SpecialCharTok{$}\NormalTok{job, }\AttributeTok{levs =} \FunctionTok{c}\NormalTok{(}\StringTok{"blue{-}collar"}\NormalTok{, }\StringTok{"technician"}\NormalTok{), }
                             \AttributeTok{newLabel =} \StringTok{"blue{-}collar"}\NormalTok{)}
\NormalTok{df\_bank}\SpecialCharTok{$}\NormalTok{job }\OtherTok{\textless{}{-}} \FunctionTok{combineLevels}\NormalTok{(df\_bank}\SpecialCharTok{$}\NormalTok{job, }\AttributeTok{levs =} \FunctionTok{c}\NormalTok{(}\StringTok{"housemaid"}\NormalTok{, }\StringTok{"services"}\NormalTok{), }
                             \AttributeTok{newLabel =} \StringTok{"service class"}\NormalTok{)}
\NormalTok{df\_bank}\SpecialCharTok{$}\NormalTok{job }\OtherTok{\textless{}{-}} \FunctionTok{combineLevels}\NormalTok{(df\_bank}\SpecialCharTok{$}\NormalTok{job, }\AttributeTok{levs =} \FunctionTok{c}\NormalTok{(}\StringTok{"entrepreneur"}\NormalTok{, }
                                                   \StringTok{"self{-}employed"}\NormalTok{), }
                             \AttributeTok{newLabel =} \StringTok{"self{-}employed"}\NormalTok{)}
\end{Highlighting}
\end{Shaded}

\textbf{Brief Conclusion}

Based on the exploratory data analysis (EDA) done, we have decided to
remove two variables from the modelling stage: pdays (since it is highly
correlated to the variable previous) and duration (since it highly
affects the target variable i.e.~if duration = 0 then y = no). Outliers
have been dealt with. For the categorical variables, some of the similar
levels have been grouped to improve their distribution. The response
variable could have been treated as well by making it more balanced by
using some advanced packages such as DMwR (function SMOTE), ROSE etc. as
a part of advanced EDA but this has not been done here.

\textbf{Part b)}

\textbf{Fit a regression model (as appropriate) using `y' as the
response variable. Model y in terms of age, job, marital and default.
Provide the model summary results and write an equation for the fitted
model on log-odds scale.}

\begin{Shaded}
\begin{Highlighting}[]
\NormalTok{df\_bank}\SpecialCharTok{$}\NormalTok{y }\OtherTok{\textless{}{-}} \FunctionTok{ifelse}\NormalTok{(df\_bank}\SpecialCharTok{$}\NormalTok{y }\SpecialCharTok{==} \StringTok{"yes"}\NormalTok{, }\DecValTok{1}\NormalTok{, }\DecValTok{0}\NormalTok{)}
\end{Highlighting}
\end{Shaded}

\begin{Shaded}
\begin{Highlighting}[]
\NormalTok{mod1 }\OtherTok{\textless{}{-}} \FunctionTok{glm}\NormalTok{(y }\SpecialCharTok{\textasciitilde{}}\NormalTok{ age }\SpecialCharTok{+}\NormalTok{ job }\SpecialCharTok{+}\NormalTok{ marital }\SpecialCharTok{+}\NormalTok{ default, }\AttributeTok{data=}\NormalTok{df\_bank,}
          \AttributeTok{family=}\FunctionTok{binomial}\NormalTok{(}\AttributeTok{link=}\StringTok{"logit"}\NormalTok{))}
\FunctionTok{summary}\NormalTok{(mod1)}
\end{Highlighting}
\end{Shaded}

\begin{verbatim}
## 
## Call:
## glm(formula = y ~ age + job + marital + default, family = binomial(link = "logit"), 
##     data = df_bank)
## 
## Deviance Residuals: 
##     Min       1Q   Median       3Q      Max  
## -0.8884  -0.5242  -0.4589  -0.3969   2.3596  
## 
## Coefficients:
##                    Estimate Std. Error z value Pr(>|z|)    
## (Intercept)       -2.212279   0.476216  -4.646 3.39e-06 ***
## age                0.012848   0.004511   2.848  0.00439 ** 
## jobNot Employed   -0.047775   0.437028  -0.109  0.91295    
## jobadmin and mgnt -0.422589   0.428048  -0.987  0.32352    
## jobblue-collar    -0.816166   0.429320  -1.901  0.05729 .  
## jobservice class  -0.730414   0.445387  -1.640  0.10102    
## jobself-employed  -0.666107   0.456582  -1.459  0.14459    
## maritalunmarried   0.467715   0.099489   4.701 2.59e-06 ***
## defaultyes         0.008619   0.361493   0.024  0.98098    
## ---
## Signif. codes:  0 '***' 0.001 '**' 0.01 '*' 0.05 '.' 0.1 ' ' 1
## 
## (Dispersion parameter for binomial family taken to be 1)
## 
##     Null deviance: 3231.0  on 4520  degrees of freedom
## Residual deviance: 3163.6  on 4512  degrees of freedom
## AIC: 3181.6
## 
## Number of Fisher Scoring iterations: 5
\end{verbatim}

\begin{Shaded}
\begin{Highlighting}[]
\FunctionTok{exp}\NormalTok{(mod1}\SpecialCharTok{$}\NormalTok{coefficients)}
\end{Highlighting}
\end{Shaded}

\begin{verbatim}
##       (Intercept)               age   jobNot Employed jobadmin and mgnt 
##         0.1094509         1.0129306         0.9533485         0.6553477 
##    jobblue-collar  jobservice class  jobself-employed  maritalunmarried 
##         0.4421236         0.4817098         0.5137046         1.5963421 
##        defaultyes 
##         1.0086559
\end{verbatim}

Fitted model equation on log-odds scale:

\(\hat{log(odds(y=1|age,job,marital,default))}\) = -2.212279 +
0.012848age - 0.047775jobNot Employed -0.422589jobadmin and mgnt
-0.816166jobblue-collar -0.730414jobservice class
-0.666107jobself-employed + 0.467715maritalunmarried +
0.008619defaultyes

\textbf{Part c)}

\textbf{Based on the model in part b), interpret the regression
coefficients for the variables age and default on an appropriate scale.}

Interpretation of regression coefficient associated with the variable
age

exp(\(\hat{\beta_1}\)) = 1.0129306. It represents the multiplicative
effect of age on the odds ratio of subscribing to a term deposit,
holding all other variables constant. It is estimated as exp(0.012848) =
1.0129306, i.e.~for every one year increase in age the odds of
subscribing to a term deposit increases by 1.3\%, keeping all other
variables unchanged.

Interpretation of regression coefficient associated with the variable
default

exp(\(\hat{\beta_8}\)) = 1.0086559. It represents the odds ratio of
subscribing to a term deposit for an individual who has defaulted in
comparison to an individual who has not defaulted, keeping all other
variables fixed. It is estimated as exp(0.008619) = 1.0086559 that is
the odds of subscribing to a term deposit are 1.0086559 times higher for
individuals who have defaulted compared to individuals who haven't
defaulted, holding all other variables unchanged.

\textbf{Part d)}

\textbf{Based on the model in part b), what is the estimated odds that
an unmarried self-employed individual who is 30 years old and has
defaulted will subscribe for the term deposit? What is the estimated
probability that a married blue-collared individual who is 50 years old
and has not defaulted will subscribe for the term deposit?}

\textbf{Estimated odds}

\(\hat{odds(y=1|Marital=unmarried, job=self-employed, age=30, default=yes)}\)
= exp(\(\hat{\beta_0}\) + \(\hat{\beta_1}\)*30 + \(\hat{\beta_6}\) +
\(\hat{\beta_8}\))

= exp(-2.212279 + 0.012848*30 - 0.666107 + 0.008619)

= exp(-2.484327)

= 0.08338165

Therefore, the estimated odds that an unmarried self-employed individual
who is 30 years old and has defaulted will subscribe for the term
deposit is 0.08338165.

\textbf{Estimated Probability}

\(\hat{P(y=1|Marital=married, job=blue-collar, age=50, default=no)}\) =
\(\frac{exp(\hat\beta_0+\hat\beta_1*50+\hat\beta_4)}{1+exp(\hat\beta_0+\hat\beta_1*50+\hat\beta_4)}\)

=
\(\frac{exp(-2.212279+0.012848*50-0.816166)}{1+exp(-2.212279+0.012848*50-0.816166)}\)

= \(\frac{exp(-2.386045)}{1+exp(-2.386045)}\)

= = \(\frac{0.0919928}{1+0.0919928}\)

= 0.08424303

Therefore, the estimated probability that a married blue-collared
individual who is 50 years old and has not defaulted will subscribe for
the term deposit is 0.08424303.

\textbf{Part e)}

\textbf{Fit another model which includes the variables age, default,
balance, marital, job, campaign and an interaction between balance and
marital. Provide the model summary results as obtained in R and write an
equation for the fitted model on odds scale.}

\begin{Shaded}
\begin{Highlighting}[]
\NormalTok{mod2 }\OtherTok{\textless{}{-}} \FunctionTok{glm}\NormalTok{(y }\SpecialCharTok{\textasciitilde{}}\NormalTok{ age}\SpecialCharTok{+}\NormalTok{ default }\SpecialCharTok{+}\NormalTok{ balance }\SpecialCharTok{+}\NormalTok{ marital }\SpecialCharTok{+}\NormalTok{ job }\SpecialCharTok{+}\NormalTok{ campaign }\SpecialCharTok{+}\NormalTok{ balance}\SpecialCharTok{*}\NormalTok{marital, }\AttributeTok{data=}\NormalTok{df\_bank,}
          \AttributeTok{family=}\FunctionTok{binomial}\NormalTok{(}\AttributeTok{link=}\StringTok{"logit"}\NormalTok{))}
\FunctionTok{summary}\NormalTok{(mod2)}
\end{Highlighting}
\end{Shaded}

\begin{verbatim}
## 
## Call:
## glm(formula = y ~ age + default + balance + marital + job + campaign + 
##     balance * marital, family = binomial(link = "logit"), data = df_bank)
## 
## Deviance Residuals: 
##     Min       1Q   Median       3Q      Max  
## -0.9731  -0.5306  -0.4500  -0.3815   2.5995  
## 
## Coefficients:
##                            Estimate Std. Error z value Pr(>|z|)    
## (Intercept)              -2.025e+00  4.837e-01  -4.187 2.83e-05 ***
## age                       1.188e-02  4.539e-03   2.617  0.00886 ** 
## defaultyes                1.206e-01  3.646e-01   0.331  0.74072    
## balance                   8.794e-05  2.978e-05   2.953  0.00314 ** 
## maritalunmarried          4.951e-01  1.208e-01   4.100 4.14e-05 ***
## jobNot Employed          -7.728e-02  4.396e-01  -0.176  0.86047    
## jobadmin and mgnt        -4.187e-01  4.306e-01  -0.972  0.33090    
## jobblue-collar           -7.980e-01  4.319e-01  -1.848  0.06463 .  
## jobservice class         -7.206e-01  4.480e-01  -1.609  0.10771    
## jobself-employed         -6.576e-01  4.591e-01  -1.432  0.15206    
## campaign                 -1.149e-01  2.775e-02  -4.141 3.46e-05 ***
## balance:maritalunmarried -2.422e-05  4.562e-05  -0.531  0.59551    
## ---
## Signif. codes:  0 '***' 0.001 '**' 0.01 '*' 0.05 '.' 0.1 ' ' 1
## 
## (Dispersion parameter for binomial family taken to be 1)
## 
##     Null deviance: 3231.0  on 4520  degrees of freedom
## Residual deviance: 3133.1  on 4509  degrees of freedom
## AIC: 3157.1
## 
## Number of Fisher Scoring iterations: 5
\end{verbatim}

\begin{Shaded}
\begin{Highlighting}[]
\FunctionTok{exp}\NormalTok{(mod2}\SpecialCharTok{$}\NormalTok{coefficients)}
\end{Highlighting}
\end{Shaded}

\begin{verbatim}
##              (Intercept)                      age               defaultyes 
##                0.1319797                1.0119515                1.1282260 
##                  balance         maritalunmarried          jobNot Employed 
##                1.0000879                1.6407274                0.9256338 
##        jobadmin and mgnt           jobblue-collar         jobservice class 
##                0.6579124                0.4502142                0.4864597 
##         jobself-employed                 campaign balance:maritalunmarried 
##                0.5180970                0.8914525                0.9999758
\end{verbatim}

Fitted model equation on odds scale:

\(\hat{odds(y=1|balance,marital,job,campaign)}\) = exp(-2.025e+00 +
1.188e-02age + 1.206e-01defaultyes + 8.794e-05balance +
4.951e-01maritalunmarried - 7.728e-02jobNot Employed - 4.187e-01jobadmin
and mgnt - 7.980e-01jobblue-collar - 7.206e-01jobservice class -
6.576e-01jobself-employed - 1.149e-01campaign -
2.422e-05balance*maritalunmarried)

\textbf{Part f)}

\textbf{Based on the model in part e), interpret the regression
coefficients with the main effects of variables balance and marital on
an appropriate scale.}

Interpretation of regression coefficient associated with the main effect
of variable balance

exp(\(\hat{\beta_3}\)) = 1.0000879. For every one dollar increase in the
balance for a married individual, the odds of subscribing to a term
deposit is multiplied by a factor of exp(8.794e-05) = 1.0000879, when
all other variables remain unchanged.

Interpretation of regression coefficient associated with the main effect
of variable marital

exp(\(\hat{\beta_4}\)) = 1.6407274. It represents the odds ratio of
subscribing to a term deposit which is multiplied by a factor of
exp(4.951e-01) = 1.6407274 for an unmarried individual who has a balance
of 0 in his/her bank account, keeping all other variables fixed.

\textbf{Part g)}

\textbf{What is the estimated odds ratio for a 25 year old married
service class individual who has a balance of 1000, has been contacted 3
times during this campaign and has defaulted vs.~a 40 year old unmarried
blue-collar individual who has a balance of 5000 dollars, has been
contacted once during this campaign and has not defaulted?}

Estimated odds ratio is calculated as follows:

\(\frac{odds(y=1|age=25, default=yes, balance=1000, marital=married, job=service class, campagin=3)}{odds(y=1|age=40, default=no, balance=5000, marital=unmarried, job=blue-collar, campagin=1)}\)

=
\(\frac{exp(\hat\beta_0+\hat\beta_1*25+\hat\beta_2+\hat\beta_3*1000+\hat\beta_8+\hat\beta_{10}*3)}{exp(\hat\beta_0+\hat\beta_1*40+\hat\beta_3*5000+\hat\beta_4+\hat\beta_7+\hat\beta_{10}*1+\hat\beta_{11}*5000*1)}\)

=
\(\frac{exp(-2.025e+00+1.188e-02*25+1.206e-01+8.794e-05*1000-7.206e-01-1.149e-01*3)}{exp(-2.025e+00+1.188e-02*40+8.794e-05*5000+4.951e-01-7.980e-01-1.149e-01*1-2.422e-05*5000*1)}\)

= \(\frac{0.07541418}{0.1922421}\)

= 0.3922875

Therefore, the estimated odds ratio for a 25 year old married service
class individual who has a balance of 1000, has been contacted 3 times
during this campaign and has defaulted vs.~a 40 year old unmarried
blue-collar individual who has a balance of 5000 dollars, has been
contacted once during this campaign and has not defaulted is 0.3922875.

\textbf{Part h)}

\textbf{Formally compare the models in parts b) and e) using analysis of
deviance. What is your conclusion? Which model will you select on the
basis of AIC and BIC?}

\begin{Shaded}
\begin{Highlighting}[]
\FunctionTok{anova}\NormalTok{(mod1,mod2,}\AttributeTok{test=}\StringTok{"Chisq"}\NormalTok{)}
\end{Highlighting}
\end{Shaded}

\begin{verbatim}
## Analysis of Deviance Table
## 
## Model 1: y ~ age + job + marital + default
## Model 2: y ~ age + default + balance + marital + job + campaign + balance * 
##     marital
##   Resid. Df Resid. Dev Df Deviance  Pr(>Chi)    
## 1      4512     3163.6                          
## 2      4509     3133.1  3   30.508 1.079e-06 ***
## ---
## Signif. codes:  0 '***' 0.001 '**' 0.01 '*' 0.05 '.' 0.1 ' ' 1
\end{verbatim}

\begin{Shaded}
\begin{Highlighting}[]
\NormalTok{mod1}\SpecialCharTok{$}\NormalTok{aic}
\end{Highlighting}
\end{Shaded}

\begin{verbatim}
## [1] 3181.603
\end{verbatim}

\begin{Shaded}
\begin{Highlighting}[]
\NormalTok{mod2}\SpecialCharTok{$}\NormalTok{aic}
\end{Highlighting}
\end{Shaded}

\begin{verbatim}
## [1] 3157.095
\end{verbatim}

\begin{Shaded}
\begin{Highlighting}[]
\FunctionTok{BIC}\NormalTok{(mod1)}
\end{Highlighting}
\end{Shaded}

\begin{verbatim}
## [1] 3239.352
\end{verbatim}

\begin{Shaded}
\begin{Highlighting}[]
\FunctionTok{BIC}\NormalTok{(mod2)}
\end{Highlighting}
\end{Shaded}

\begin{verbatim}
## [1] 3234.093
\end{verbatim}

From the output above, our simplified model is Model 1 which includes 4
variables: age, job, marital and default. We want to test if Model 1 is
an adequate simplification of the complete model (Model 2) which apart
from these variables also includes balance, campaign and an interaction
between balance and marital.

Hypothesis is defined as follows:

\({H_0}\) : \({\beta_3}\) = \({\beta_{10}}\) = \({\beta_{11}}\) = 0

\({H_1}\) : at least one of \({\beta_3}\), \({\beta_{10}}\),
\({\beta_{11}}\) \(\neq\) 0

Deviance = 30.508

p-value = 1.079e-06

Since the p-value is much lower than any reasonable \(\alpha\), we can
reject the null hypothesis and conclude that Model 1 is NOT an adequate
simplification of the complete model (Model 2).

Since both AIC (3157.095 \textless{} 3181.603) and BIC (3234.093
\textless{} 3239.352) values are lower for Model 2, therefore Model 2
should be selected.

\end{document}
